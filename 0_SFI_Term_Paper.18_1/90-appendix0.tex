\chapter{Некоторые из молитв «образа» (258--263), изложенные в Евхологии Барберини св. Марка}\label{ux43dux435ux43aux43eux442ux43eux440ux44bux435-ux438ux437-ux448ux435ux441ux442ux438-ux43cux43eux43bux438ux442ux432-ux43e-ux43cux43eux43dux430ux445ux438ux43dux435-258263-ux438ux437ux43bux43eux436ux435ux43dux43dux44bux435-ux432-ux435ux432ux445ux43eux43bux43eux433ux438ux438-ux431ux430ux440ux431ux435ux440ux438ux43dux438-ux441ux432.-ux43cux430ux440ux43aux430}

Здесь изложены четыре из шести молитв Евхология Барберини св. Марка\cite[С.219–222]{barberini.evhologiy.2011}, наиболее схожие с текстами антифонов чина пострижения черниц ангельского образа и, вероятно, послужившие основой для них.

\textbf{{[}258{]}} Молитва о принимающей образ монашеский.\\
Господи Боже наш, так возлюбивший девственность, что назначил матерью твоего домостроительства, сам, Владыкa всяческих, прими твою рабыню \emph{имярек}, восхотевшую оставить эту мирскую и преходящую славу и созерцать твою благость, и удостой ее посещения твоего Духа Святого, даруй ей воздержание и всяческую скромность чтобы, живя угодно тебе, как сияние благоразумия несла светильник, непрестанно исполняясь елеем своих дел, навстречу тебe, жениху небесному. Ибо благословенно и прославлено (твое имя) достойное хвалы.

\textbf{{[}259{]}} Вторая молитва о монахине.\\
Боже благ, Отец девственности и возделыватель добродетельной жизни, сохрани эту рабыню твою в красоте избранной, облачи ее сиянием святости и одеждой целомудрия, чтобы заботилась о Господнем, как угодить Владыке, чтобы была святой в телe, в душе и в духе, и, сопричисляясь к сонму разумных дев и неся светильник благочестия, удостоилась войти в небесные чертоги жизни вечной. Во Христе Иисусе Господе нашем, с которым ты благословен, со пресвятым и благим и животворящим.

\textbf{{[}260{]}} Третья молитва образа.
Господи Боже нашa, призывающий праведников к святости и грешников к оправданию, и обращающихся быть помилованными, сам, Господи человеколюбец, прими обращение твоей рабыни \emph{имярек}, и по благости прости ее, припадающую перед твоим милосердием в исповедании своих грехов. Искупи ее от всякого греха, да воссияет над всяким нечистым помышлением, очисти ее от всякой скверны плоти и духа и укрепи в исполнении заповедей твоих, сняв с нее суету мирскую и отрезав ей волосы в подражание твоей первомученице Фекле; сделай ее сопричисленной к верным и разумным девам, чтобы в чистоте души и тела принoся добрый плод и возрастая в добрых делах, стала наследницей царства небесного. Во Христе Иисусе Господe нашем с которым ты благословен, с пресвятым и благим.

\textbf{{[}262{]}} Пятая молитва образа.
Свят, Свят, обитающий в святыхa, хотящий, чтобы все люди спас- лись и пришли в познание истиныb, так возлюбивший род челове- ческий что вселился и пребывал в немc чтобы он был храмoм твоей славы страшной и бессмертной, ты сказал: если кто хочет идти за мною, да отвергнется от себя, и возьмeт крест свой, и последует за мноюd, сам, Владыка человеколюбец, благоволи к отречению твоей рабыни имярек, наложи на нее иго твое благое и бремя твое легкоеe, сохрани цельными ее Дух, душу и телоf, облачи ее в одежду первуюg духовную и нетленную, одень ее броней верыh и всеоружием Духа Святогоi, препоясай ее чистотой, святостью, целомудриемj, чтобы укрепившись твоей силой против козней дьявольскихk, подвизаться подвигом добрым, совершить твой путь, сохранить твою веруl и получить венец праведностиm, который ты обещал любящим тебяn и соблюдающим твои заповеди. Ибо подобает тебе всякая славаo, честь и поклонение.
