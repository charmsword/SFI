\chapter{Проповеди язычникам}
\label{cha:appendix2}

\subsection*{Проповедь ап Павла в Листрах (Деян 14:8-18}
8 И некий муж в Листрах, не владевший ногами, сидел; хромой от чрева матери своей, он никогда не ходил.

9 Он слышал, как говорил Павел, который, устремив на него взор и увидев, что он имеет веру, чтобы быть спасенным,

10 сказал громким голосом: встань на ноги твои прямо. И он вскочил и стал ходить.

11 Народ же, увидев, что сделал Павел, возвысил свой голос, говоря по-ликаонски: боги в образе человеческом сошли к нам.

12 И называли они Варнаву Зевсом, а Павла Гермесом, так как он держал речь.

13 И жрец Зевса, стоящего перед городом, доставив к воротам быков и венки, хотел с народом принести жертву.

14 Но апостолы Варнава и Павел, услышав, разорвали одежды свои и с криком бросились в толпу,

15 говоря: мужи, что это вы делаете? И мы – подобные вам люди, благовествующие вам, чтобы вы от этих суетных богов обратились к Богу живому, Который сотворил небо и землю, и море и всё, что в них,

16 Который в прошедших поколениях позволил всем народам ходить своими путями,

17 хотя и не переставал свидетельствовать о Себе, творя добро, подавая вам с неба дожди и времена плодоносные, исполняя пищею и радостью сердца ваши.

18 И говоря это, они едва успокоили народ, чтобы не приносили им жертвы.

\subsection*{Проповедь ап Павла в Ареопаге (Деян 17:16-34)}
16 И пока Павел ожидал их в Афинах, дух его в нем возмущался, видя, что город полон идолов.

17 Итак, он рассуждал в синагоге с Иудеями и чтущими Бога, и на площади каждый день со случайными встречными.

18 А кое-кто и из эпикурейских и стоических философов встречался с ним, и некоторые говорили: что хочет сказать этот пустослов? А другие: кажется, это проповедник чужих богов (потому что он благовествовал Иисуса и воскресение).

19 И взяв его, привели в Ареопаг и говорили: можем ли мы узнать, что это за новое учение, проповедуемое тобой?

20 Ибо странное что-то вкладываешь ты в наши уши. Вот мы и хотим узнать, что это может быть?

21 Афиняне же все и живущие у них чужестранцы ничем другим не заполняли свои досуги, как тем, чтобы говорить или слушать что-нибудь новое.

22 И Павел, став посредине Ареопага, сказал: мужи Афиняне, по всему вижу, что вы особенно богобоязненны.

23 Ибо, проходя и осматривая ваши святыни, я нашел и жертвенник, на котором было написано: "неведомому богу". Итак, что вы, не зная, чтите, я возвещаю это вам.

24 Бог, сотворивший мир и всё, что в нём, Он, Господь неба и земли, не в рукотворенных храмах обитает,

25 и не руками человеческими воздается Ему служение, как имеющему в чем-либо нужду: Он Сам дарует всем жизнь и дыхание и всё.

26 И произвёл Он от одного весь род человеческий: обитать по всему лицу земли, предуставив сроки и пределы их обитанию;

27 искать Бога, не коснутся ли они Его и не найдут ли, хотя и не далеко Он от каждого из нас.

28 Ибо в Нем мы живем и движемся и существуем, как и некоторые из ваших поэтов сказали: "Ведь мы Его и род".

29 Итак, будучи родом Божиим, мы не должны думать, что Божество подобно золоту, или серебру, или камню, носящим печать искусства и мысли человеческой,

30 Поэтому, оставив без внимания времена неведения, Бог теперь возвещает людям, всем и всюду, чтобы они каялись,

31 ибо Он определил день, когда будет судить вселенную по праведности, чрез Мужа, Которого Он поставил, дав удостоверение всем, воскресив Его из мертвых.

32 Услышав же о воскресении мертвых, одни насмехались, другие сказали: мы послушаем тебя об этом еще раз.

33 Так Павел вышел из среды их.

34 Но некоторые люди, примкнув к нему, уверовали: между ними и Дионисий Ареопагит и женщина, по имени Дамарь, и другие с ними.

\subsection*{Проповедь ап Павла прокуратору Феликсу (Деян 24:10-26)}
10 И ответил Павел, когда правитель сделал ему знак говорить: зная, что ты с давних лет судишь этот народ, я с легким сердцем буду защищать мое дело,

11 так как ты можешь узнать, что – не более двенадцати дней, как я пришел для поклонения в Иерусалим.

12 И не нашли меня ни в храме с кем-либо спорящим или вызывающим волнение народа, ни в синагогах, ни в городе,

13 и не могут доказать тебе того, в чем теперь обвиняют меня.

14 Но в том я признаюсь тебе, что на Пути, который они называют ересью, я, действительно, служу Богу отцов наших, веруя всему, что согласно с Законом и что написано у Пророков,

15 надежду имея на Бога, которую и сами они разделяют, – что будет воскресение и праведных и неправедных.

16 Потому я и сам стараюсь всегда иметь непорочную совесть пред Богом и людьми.

17 Через несколько лет я прибыл с целью передать милостыни народу моему и приношения.

18 При этом нашли меня очистившимся в храме, не с толпой и не с шумом,

19 нашли же какие-то Иудеи из Асии, которым надлежало бы быть здесь у тебя и обвинять, если бы у них было что против меня.

20 Или пусть они сами скажут, какое нашли они преступление, когда я предстал пред синедрионом,

21 кроме одного этого слова, которое я возгласил, стоя между ними: "за воскресение мертвых вы меня судите сегодня".

22 Но Феликс, имея более точные сведения о Пути, отослал их для дополнительного расследования, сказав: когда придет трибун Лисий, я рассмотрю ваше дело;

23 и дал распоряжение сотнику держать его под стражей, но давать ему некоторые послабления, и не препятствовать никому из близких его служить ему.

24 Несколько дней спустя, Феликс, прибыв с Друзиллой, женой своей, Иудеянкой, послал за Павлом и слушал его о вере во Христа Иисуса.

25 Но так как он говорил о праведности и обладании собой и о будущем суде, то Феликс, придя в страх, ответил: пока иди, а при случае я тебя вызову к себе.

26 В то же время он и надеялся, что Павел даст ему денег. Потому-то, часто за ним посылая, он и беседовал с ним.

\subsection*{Проповедь ап Павла перед царём Агриппой (Деян 26:1-31)}
1 И сказал Агриппа Павлу: тебе разрешается говорить за себя. Тогда Павел, протянув руку, начал свою защитительную речь:

2 во всём, в чем обвиняют меня Иудеи, царь Агриппа, я почитаю себя счастливым защищаться сегодня перед тобой,

3 потому особенно, что ты знаток всех обычаев и спорных мнений Иудеев. Поэтому прошу выслушать меня с долготерпением.

4 Жизнь мою от юности, протекавшую с самого начала среди народа моего в Иерусалиме, ведают все Иудеи,

5 зная обо мне издавна, если есть у них желание свидетельствовать – что жил я фарисеем, по строжайшему направлению в нашей вере.

6 И теперь я стою перед судом за надежду на обещание, бывшее от Бога отцам нашим,

7 исполнения которого надеются достичь наши двенадцать колен, усердно служа Богу день и ночь. За эту надежду, царь, и обвиняют меня Иудеи.

8 Почему у вас считается невероятным, что Бог воздвигает мертвых?

9 Ведь я и сам думал, что против имени Иисуса Назорея надо мне многое сделать.

10 Это я и делал в Иерусалиме, и многих из святых я заключил в тюрьмы, получив власть от первосвященников, и когда их убивали, подавал свой голос против них;

11 и по всем синагогам, многократно наказывая их, принуждал к хуле и, в чрезмерной против них ярости, преследовал их даже и в чужих городах.

12 В этих условиях, отправляясь в Дамаск с полномочиями и поручением от первосвященников,

13 я в полдень на дороге увидел, царь, свет с неба, сильнее солнечного блеска, осиявший меня и шедших со мной.

14 И когда все мы упали на землю, я услышал голос, говорящий мне на еврейском языке: "Саул, Саул, что ты меня гонишь? Трудно тебе идти против рожна".

15 Я сказал: "Кто Ты, Господи?" Господь сказал: "Я Иисус, Которого ты гонишь.

16 Но встань и стань на ноги твои, ибо Я для того явился тебе, чтобы поставить тебя служителем и свидетелем Моим, как ты Меня видел, и как Я явлюсь тебе,

17 избавляя тебя от народа и от язычников, к которым Я посылаю тебя,

18 открыть им глаза, чтобы обратились они от тьмы к свету и от власти сатаны к Богу, и чтобы получили они отпущение грехов и удел вместе с освященными, по вере в Меня".

19 Поэтому, царь Агриппа, я не оказал непослушания небесному видению,

20 но сперва находящимся в Дамаске и Иерусалиме, и по всей стране Иудейской, и язычникам возвещал, чтобы они каялись и обращались к Богу, творя дела, достойные покаяния.

21 За это Иудеи, задержав меня в храме, пытались расправиться со мной.

22 Итак, с помощью, которая приходит от Бога, я устоял до сего дня, свидетельствуя малому и великому, не говоря ничего, кроме того, о чем Пророки сказали и Моисей, что должно тому быть:

23 Христу – пострадать и, воскресши первым из мертвых, свет возвещать народу и язычникам.

24 Когда он так защищался, Фест громким голосом говорит: ты безумствуешь, Павел! Большая ученость доводит тебя до безумия.

25 А Павел говорит: я не безумствую, превосходнейший Фест, но возглашаю слова истины и здравого смысла.

26 Ибо знает об этом царь, которому я и говорю с дерзновением. Ведь я не верю, чтобы сокрыто было от него что-либо из этого; ибо не в углу это было совершено.

27 Веришь ли, царь Агриппа, Пророкам? Знаю, что веришь.

28 Но Агриппа Павлу: еще немного, и ты будешь убеждать меня сделаться христианином.

29 А Павел: молил бы я Бога, чтобы мало ли, много ли, не только ты, но и все слушающие меня сегодня, сделались такими же как и я, кроме этих уз.

30 И встал царь и правитель и Вереника и сидевшие с ними,

31 и удалившись, говорили между собой, что ничего достойного смерти или уз человек этот не делает.

%%% Local Variables: 
%%% mode: latex
%%% TeX-master: "rpz"
%%% End: 

