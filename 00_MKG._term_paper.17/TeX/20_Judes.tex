
\chapter{ Особенности миссионерской проповеди иудеям и богобоязненным}
\label{cha:judes}
%
% % В начале раздела  можно напомнить его цель
%
Среди иудеев были не только евреи, но и обратившиеся в иудаизм из других народов: прозелиты, богобоязненные (не обрезавшиеся, но соблюдавшие Закон). Потому мы рассматриваем в качестве миссии иудеям проповедь ап Филиппа эфиопскому евнуху и проповедь ап Петра сотнику Корнилию, "мужу праведному и боящемуся Бога" с домочадцами.

\subsubsection*{Проповедь ап Петра в день Пятидесятницы (Деян 2:14-2:47)}
%%внешний контекст
После сошествия Св Духа на апостолов, они "начали говорить иными языками" (2:4), что привлекло внимание окружающих.
Так как был большой праздник Шавуот, в Иерусалиме присутствовало множество народу, как местных, так и приезжих "Иудеев, благоговейных людей из всякого народа под небом".
Слух о странном событии, случившимся с апостолами, собрал толпу, "пришедших в смятение", "изумлявшихся" и "дивившихся", т.к. каждый слышал, что апостолы говорили на его наречье.
К этим разноплемённым, но исповедавшим одну ветхозаветную веру людям, вышел со словом проповеди апостол Пётр.

Браун называет эту проповедь "фундаментальной формулировкой благовестия" в Деян (\cite{@brown.vvedenie_1}, 10.2.2).
В ней формируется керигма апостольской проповеди, отличной от проповеди Иисуса христологическим содержанием и исповеданием Иисуса воскресшим, Господом, Мессией, Сыном Божьим. 

%%структура проповеди и внутр. логика
\begin{center}
	\begin{longtable}{ |c|c|p{0.70\textwidth}| } 
 \hline
 стих & смысл & содержание \\ 
 \hline\hline
 2:16-21 & напоминание & пророчества Иоиля о сошествии Духа \\ 
 2:22-24 & керигма & предведение Божье об Иисусе, убийство и воскресение \\ 
 2:25-28 & напоминание & пророчество царя Давида о воскресении \\
 2:29-31 & толкование & Давид говорил о Христе \\
 2:32 & керигма & "Этого Иисуса воскресил Бог, чему все мы свидетели" \\
 2:33 & керигма & Иисус излил Духа на апостолов \\
 2:34-35 & напоминание & Давид пророчествовал о Господе \\
 2:36 & керигма & обличение: дом Израилев распял Господа и Христа \\
 2:38 & призыв & покатесь, креститесь, примите дар Святого Духа \\
 2:39 & керигма & обещание Божье – для всех, в т.ч. "дальних" \\ 
 \hline
\end{longtable}
\end{center}


В ходе проповеди, люди начали задавать вопросы ("Что нам делать, мужи братья?" 2:37), и возможно, вопросы не прекратились после, так как и "иными многими словами" апостол "свидетельствовал и увещал их".
И многие ("около трёх тысяч") крестились и вошли в общение и учение апостолов.

%%керигма
Керигма проповеди ап Петра: 
\begin{itemize}
 \item исповедание Иисуса Христом, о котором "предвидел Бог" и говорили пророки
 \item Иисуса воскресил Бог
 \item Иисус принял Духа от Бога и ниспослал Его на апостолов
 \item Иисус был распят
 \item Благая Весть теперь – для всех
\end{itemize}

Каждое возвещение апостол сопровождает ссылкой на Ветхий Завет: о Мессии и воскресении пророчествовал пророк Давид, о сошествии Духа – пророк Иоиль, обещание для "дальних" напоминает Ис 57:19,
а обличение "дома Израилева" в распятии Христа созвучно пророческим обличениям.
Апостол цитирует писание пророков и истолковывает, как пророчества раскрываются в Иисусе Христе.
Кроме того, что это подкрепляет его речь авторитетом Писания, а в слушателях – пробуждает память об уже воспринятом откровении,
толкование ап Петра раскрывает ранее непонятные и смутные смыслы о преодолении "тления" и наступлении "того Дня".
Он толкует то, о чём никто не мог говорить уверенно, и при том возвещает, что происходит это сейчас, у всех на глазах.

%%св-во о ч-ке и мире
Свидетельство о мире и человеке принимает образ эсхатологического исполнения: как в "тот День" (Иоиль 3:18), люди примут дар Святого Духа и получит отпущение грехов.
%% призыв к Покаянию
Это предваряется условием: покаянием и крещением (погружением) "во имя Иисуса Христа".

\subsubsection*{Проповедь апп Петра и Иоанна по исцелению хромого у Храма (Деян 3:1-4:4)}
%%внешний контекст
Апостолы продолжали ходить в храм для молитвы и теперь также для проповеди.
На ступенях они повстречали калеку, который не мог быть в храме из-за телесного изъяна, но просил милостыню у ворот.
В ответ на просьбу о милостыне, ап Пётр исцелил его от хромоты, и тот вошёл в храм, "ходя и скача и хваля Бога".
Бывшие в храме люди "исполнились ужаса и изумления" и "сбежался к ним (апостолам и исцелённому) в трепете весь народ в притворе, называемом Соломоновым".
К этим благочестивым иудеям обратил проповедь ап Пётр.

%%структура проповеди и внутр. логика
\begin{center}
	\begin{longtable}{ |c|c|p{0.55\textwidth}| } 
		\hline
		 стих & смысл & содержание \\
		   \hline\hline
		   3:12 & обращение & это чудо – не собственной силой апостолов \\
		   3:13a & керигма & Бог прославил Иисуса, "Отрока Своего" \\
		     3:13b-15a & керигма & вы отреклись и убили Начальника жизни \\
		     3:15b & керигма & Бог воздвиг Иисуса из мёртвых \\
		     3:16 & свидетельство & вера во имя Иисуса исцелила хромого \\
		      3:17 & обращение & вы отреклись по неведению \\
		      3:18 & напоминание & пророки предрекли страдания Христа \\
		       3:19-20 & призыв & "покайтесь и обратитесь" \\
		       3:21 & керигма, напоминание & Иисус на небе "до времён восстановления всего" \\
		       3:22-23 & напоминание & Моисей обещал о воздвижении Пророка "из братьев ваших" \\
		       3:24 & напоминание & Самуил и другие пророки "возвестили эти дни" \\
		       3:25 & напоминание & "вы сыны пророков и завета" \\
		       3:26 & керигма & Бог воскресил Иисуса и послал Его к каждому из вас, чтобы "отвратить от злых дел ваших" \\
		\hline
	\end{longtable}
\end{center}

%%керигма
Керигма проповеди апостола Петра:
\begin{itemize}
	\item Бог прославил Иисуса
	\item Иисус – "отрок" Божий
	\item Иисус был убит
	\item Бог воздвиг Иисуса из мёртвых
	\item Иисус придёт во "времена восстановления всего"
	\item Иисус послан к каждому
\end{itemize}

Помимо возвещения, апостол свидетельствует о чуде исцеления хромого "верой во имя Иисуса", обращает несколько раз внимание слушателей на самих себя, "мужей израильских", "братьев", к которым обращён призыв "Бога отцов ваших", и вспоминает пророчество Моисея, Самуила и "других пророков" об "этих днях" чудес.

%%св-во о ч-ке и мире
Свидетельство о мире и человеке опять звучит в эсхатологическом ключе: дни, о которых сказано у пророков, наступили, и, следовательно, чудеса не для изумления, а для "покаяния и обращения" к Сыну Божьему, о Котором все пророчества.

%% призыв к Покаянию
Апостол не успевает завершить проповедь призывом, так как "пока они говорили к народу, приступили к ним... и наложили на них руки и отдали под стражу" (4:1-3)
Однако, в 3:19-20 уже прозвучал призыв к покаянию, в 3:23 апостол предупреждает не слушающегося "Пророка Того", а в 3:26 говорит, что "Отрок" Божий послан Богом, "чтобы каждого отвратить от злых дел ваших".
Многих свидетельство исцеления и слово проповеди привело в Церковь (3:4).  


\subsubsection*{Проповедь апп Петра и Иоанна перед первосвященниками (Деян 4:5-21; 5:27-33)}
%%внешний контекст
После того, как апостолы были схвачены, собрался синедрион для решения по их вопросу, и их стали доправшивать, "какой силой или каким именем вы это сделали?".
Апостол Пётр, "исполнившись Духа Святого", начал проповедь перед первосвященниками, старейшинами и книжниками Израиля.


Увидев, что апостолов не переубедить, их изгнали с угрозами.
На следующий день они были вновь схвачены учащими в храме, и были подвергнуты новому допросу.
Апостол Пётр продолжил проповедь.

%%структура проповеди и внутр. логика
\begin{center}
	\begin{longtable}{ |c|c|p{0.70\textwidth}| } 
		\hline
		стих & смысл & содержание \\
		\hline\hline
		4:8-9 & обращение & вы допрашивайте тех, кто спас человека \\
		4:10 & свидетельство & хромой исцелён именем Иисуса Христа \\
		4:10b & керигма & Бог воздвиг Иисуса из мёртвых \\
		4:11-12 & керигма & вне Иисуса нет спасения \\
		4:19 & свидетельство & апостолы не могут не слушать Бога и не говорить о том, "что видели и слышали" \\
		5:29 & свидетельство & смысловой повтор 4:19 \\
		5:30 & обращение & обличение: вы умертвили Иисуса \\
		5:30-31 & керигма & Бог воздвиг Иисуса из мёртвых \\
		5:31b & керигма & Иисус – Начальник и Спаситель, в Котором возможно покаяние и отпущение грехов Израиля \\
		5:32 & керигма & Бог дал Духа Святого "повинующимся Ему" \\
		\hline
	\end{longtable}
\end{center}

Апостол Пётр в каждое обращение к слушающим вкладывает обличение: это они распяли Иисуса, они нуждаются в покаянии.
То, что они "дети Авраама", а Бог – "Бог их отцов", только усугбляет ситуацию, так как они погубили Того, Кого избрал и воздвиг от смерти Бог.



%%керигма
Керигма проповеди апостола Петра:
\begin{itemize}
	\item Бог воздвиг Иисуса из мёртвых
	\item вне Иисуса нет спасения
	\item в Иисусе возможно покаяние и отпущение грехов
	\item Бог даёт Духа "повинующимся Ему"
\end{itemize}

%%св-во о ч-ке и мире
Свидетельство о человеке и мире раскрывается в Иисусе: человеку необходимо обрести спасение, которое есть только "в имени Иисуса".

%% призыв к Покаянию
В слове апостола нет прямого призыва к обращению, он завуалирован в параллели между словами обличения и возвещением покаяния и отпущения грехов в Иисусе, Спасителе.

По итогам проповеди, апостолы едва не были осуждены на смерть, но по ходатайству фарисея и уважаемого законоучителя Гамалиила, были отпущены после бичевания.


\subsubsection*{Проповедь Стефана (Деян 7:2-60)}
%%внешний контекст
Дьякон Стефан, "полный благодати и силы", стал известен благодаря чудесам и знамениям.
Когда члены синагоги либертинцев, киринейцев и александрийцев затеяли с ним спор, им не удалось взять верх, и были найдены лжесвидетели, оклеветавшие юношу.
Перед судом Синедриона, после обвинений в хуле на Закон и Моисея, Стефану было дано слово и он начал проповедовать.
Браун упоминает, что ни одна из речей апостолов в Деян не приводится столь тщательно разработанной (в рамках написания текста) как речь Стефана (\cite{@brown.vvedenie_1}, 10.2.2).
В ней настолько "нестандартное понимание Ветхого Завета", что она стала предметом споров о внешнем влиянии на иудейскую традицию (там же).

%%структура проповеди и внутр. логика
\begin{center}
	\begin{longtable}{ |c|c|p{0.55\textwidth}| } 
		\hline
		стих & смысл & содержание \\
		\hline\hline
		7:2-47 & напоминание, обращение & напоминание действия Бога в Священной истории от Авраама до Соломона \\
		7:37 & напоминание & пророчество Моисея о воздвижении "пророка из братьев ваших, как меня" \\
		7:39 & обращение & обличение: "<Моисею> не захотели покориться отцы ваши" \\
		7:48-50 & напоминание & Ис 66:1, нет рукотворного дома для Бога Вседержителя \\
		7:51-53 & обращение & обличение: "вы всегда Духу Святому противитесь, как отцы ваши" \\
		7:55-56 & керигма, свидетельство & небеса отверсты, Сын Человеческий стоит по правую сторону Бога \\
		7:59 & свидетельство & молитва: "Господи Иисусе, прими дух мой" \\
		7:60 & свидетельство & ходатайственная молитва за убивающих его \\
		\hline
	\end{longtable}
\end{center}

%%керигма
Керигма проповеди Стефана:
\begin{itemize}
	\item небеса отверсты (что также может быть свидетельством о собственном пророческом даре от Духа
	\item Сын Человеческий стоит по правую сторону Бога
\end{itemize}

В проповеди Стефан, как делал Сам Иисус, не говорит прямо о Нём, но ясно даёт Его образ в служении Моисея "который принял слова живые, чтобы дать вам".
Он обличает слушателей в отвержении Моисея "которому не захотели покоиться отцы ваши" и всего обетования Божьего, данного ещё Аврааму.
А уже в этой борьбе с пророками совершается отвержение "Праведного, Которого предателями и убийцами вы теперь сделались".

%%св-во о ч-ке и мире
Стефан свидетельствует как пророк, его слова об "отверстых небесах" имеет параллель в пророческом тексте: "отверзлись небеса, и я видел видения Божии" (Иез 1:1, Син.).
А параллель с Дан 7:13-14 ("с облаками небесными шёл как бы Сын Человеческий... и Ему дана власть, слава и царство... и Царство Его не разрушится") приводит к мысли о свершившимся обетовании Бога, наступлении конца времени и Царства Бога.
Стефан свидетельствует о человеке и мире, что наступило время последнего суда и последнего выбора человека быть с Богом или отвергнуть Его.
Небо уже открыто и ждать какого-то ещё исполнения пророчеств не следует.

%% призыв к Покаянию
В словах Стефана нет прямого призыва к Покаянию, подобного призывам в проповедях ап Петра.
Есть сочетание обличения, свидетельства об исполнении обетования Бога и личная ходатайственная молитва за своих слушателей и губителей, что можно назвать личным покаянием за их грехи.

Прямое следствие проповеди – мученическая смерть Стефана.
Но, существует мнение, что его свидетельство сыграло значительную роль в истории Савла.
Так, Браун пишет "как смерть Иисуса не была Его концом, поскольку апостолы обрели Его Дух... так и смерть Стефана не была его концом, ибо её видит юноша по имени Савл... ему суждено продолжить дело Стефана" (\cite{@brown.vvedenie_1}, 10.2.2).


\subsubsection*{Проповедь ап Филиппа эфиопскому евнуху (Деян 8:26-40)}
%%внешний контекст
Апостол Филипп по повелению ангела идёт по пустынной дороге "от Иерусалима к Газе", и встречает Эфиопа, евнуха, вельможу Эфиопской царицы.
Эфиоп читал пророка Исайю и был на поклонении в Иерусалиме, откуда возвращался.
По велению ангела, ап Филипп начинает с ним разговор, после которого он крестится во имя Иисуса.

%%структура проповеди и внутр. логика
\begin{center}
	\begin{longtable}{ |c|c|p{0.55\textwidth}| } 
		\hline
		стих & смысл & содержание \\
		\hline\hline
		8:31 & обращение & понимает ли Эфиоп, что читает из пророка Исайи \\
		8:35 & напоминание, керигма & "начав от этого писания, благовествовал ему Иисуса" \\
		8:37 & керигма & евнух может креститься, если верует от всего сердца \\
		\hline
	\end{longtable}
\end{center}

%%керигма
В тексте Деян не приведена проповедь ап Филиппа, поэтому мы не можем сказать точно о её керигматической части.
Как минимум, это слово об исполнении писаний и пророчеств в Иисусе, а также универсальность откровения Нового Завета: возможность и необходимость крещения в Иисуса для каждого верующего "от всего сердца", независимо от национальных или физиологических предпосылок (ср. с запретом во Втор 23:1).
В этой универсальности – и новое откровение о человеке и мире: наступило время исполнения пророчеств, и каждый может и должен обратиться сердцем к Богу и сделать конкретный шаг: креститься. \footnote{У пророков уже звучало это сочетание эсхатологического времени и возможности иноплеменника и евнуха войти в дом и храм Божий (ср. Ис 56:3-5)}


По итогу проповеди, евнух сам спрашивает о возможности креститься и принимает крещение от ап Филиппа.
После ухода Филиппа, он "продолжал, радуясь, свой путь".

\subsubsection*{Проповедь ап Петра сотнику Корнилию с домочадцами и друзьями (Деян 10:24-48)}
%%внешний контекст
Эта проповедь продолжает "изменение правил", заложенное в крещении евнуха: Божье слово обращено к людям не из народа Израиля.
Хотя речь идёт о людях, знающих Закон, а именно "благочестивом и боящимся Бога со всем домом своим... молящимся Богу постоянно" сотнике Корнилии, они по-прежнему имеют статус "внешних".
Это оказывается столь сильным препятствием, что и для сотника, и для ап Петра требуется прямое Божье откровение, чтобы сделать шаг навстречу друг другу.

Получив видение от Бога о снятии запрета на нечистую пищу, ап Пётр преодолевает свои сомнения и приходит вслед за вестниками от Корнилия в его дом.
Корнилий созвал "родственников своих и ближайших друзей", чтобы "послушать речей" апостола

%%структура проповеди и внутр. логика
\begin{center}
	\begin{longtable}{ |c|c|p{0.55\textwidth}| } 
		\hline
		стих & смысл & содержание \\
		\hline\hline
		10:26 & свидетельство & "сам я тоже человек" \\
		10:28-29 & свидетельство & Бог указал не называть человека нечистым, "потому я пришёл" \\
		10:34-35 & керигма & Бог обращается к каждому "боящемуся Его и делающему правду" \\
		10:36 & керигма & Иисус Христос – "Господь всех" \\
		10:37-38 & напоминание & проповедь Иоанна Крестителя, жизнь Иисуса из Назарета \\
		10:38 & керигма & Иисус исполнял пророчества "благотворя и исцеляя", "Бог был с Ним" \\
		10:39 & керигма & Иисус был убит на кресте \\
		10:40-41 & керигма & Бог воздвиг Иисуса из мёртвых \\
		10:39-42 & свидетельство & мы – свидетели Его проповеди и воскресения \\
		10:42 & керигма & Иисус – "поставленный Богом Судия живых и мёртвых" \\
		10:43 & керигма, напоминание & свидетельство пророков – о Нём; в Нём – отпущение грехов верующим \\
		10:47 & свидетельство & Дух Святой сошёл на язычников, а значит и в крещении им нельзя отказать \\
		10:48 & призыв & повеление креститься \\
		\hline
	\end{longtable}
\end{center}

%%керигма
Керигма проповеди ап Петра:
\begin{itemize}
	\item Божье откровение универсально и обращено к каждому "боящемуся Его и делающему правду"
	\item Иисус Христос – Господь
	\item в Иисусе исполнились пророчества
	\item Иисус был убит на кресте
	\item Бог воздвиг Иисуса из мёртвых
	\item Иисус – Судия живых и мёртвых
	\item в Иисусе – отпущение грехов и спасение человека
	\item теперь Дух Святой сходит на людей, даже не принадлежащих народу Израиля по крови
\end{itemize}

%%св-во о ч-ке и мире
Апостол Пётр приносит новое свидетельство о человеке, основанное на собственном, новом для себя откровении.
Человек не может быть назван "нечистым" и не может быть отвергнут Богом по какому-либо внешнему признаку.
Дух Святой ниспослан Богом для каждого верующего во имя Иисуса.
Помимо факта прихода к "язычникам" и свидетельства о сошествии на них Духа, ап Пётр говорит о себе "сам я тоже человек", никак не выделяя ни своего апостольства, ни принадлежности израильскому народу.

Ещё одно важное свидетельство – Иисус – "Судия живых и мёртвых", т.е. свидетельство о Суде, на который предстанет (или уже предстаёт) каждый человек.

%% призыв к Покаянию
Призыв апостола креститься звучит уже в ответ на сошествие Духа Святого на Корнилия с друзьями и домочадцами.

\subsubsection*{Проповедь ап Павла в Антиохии Писидийской (Деян 13:14-44)}
%%внешний контекст
Апостол Павел со спутниками по повелению Духа Святого, путешествовали и проповедовали в Иудейских синагогах.
В Писидийской Антиохии в синагоге их пригласили сказать "слово наставления к народу" после чтения Закона и Пророков, и ап Павел стал говорить.

%%структура проповеди и внутр. логика
\begin{center}
	\begin{longtable}{ |c|c|p{0.70\textwidth}| } 
		\hline
		стих & смысл & содержание \\
		\hline\hline
		13:16 & обращение & "мужи Израильские и боящиеся Бога, послушайте" \\
		13:17-22 & напоминание & рассказ о Священной истории от Египетского плена до царя Давида \\
		13:23 & керигма & к Израилю пришёл Иисус Спаситель, сын Давида \\
		13:24-25 & напоминание & свидетельство Иоанна Крестителя об Иисусе \\
		13:26 & обращение & "нам послано слово спасения этого" \\
		13:27-29 & керигма & Иисус был отвергнут "начальниками" и "живущими в Иерусалиме" и распят \\
		13:30 & керигма & Бог воздвиг Иисуса из мёртвых \\
		13:31 & свидетельство & мы и многие другие – свидетели Его воскресения \\
		13:32-37 & керигма & в Иисусе исполняются пророчества \\
		13:33-35 & напоминание & Пс 2:7, "милости, обещанные Давиду" (ср. Иез 37:24, Ос 3:5), Пс 88:49 \\
		13:38-39 & керигма & во Христе – отпущение грехов и Спасение, которого не может быть в Законе \\
		13:40-41 & призыв & покайтесь ("презрители, подивитесь и сгиньте") \\
		13:41 & керигма & наступил день, когда "дело делаю Я (Господь)" \\
		\hline
	\end{longtable}
\end{center}

%%керигма
Керигма апостола Павла:
\begin{itemize}
	\item пришёл Иисус, Спаситель
	\item Иисус – сын Давида
	\item в Иисусе сбываются пророчества
	\item наступили дни Господа, возвещённые в пророчествах
	\item Иисуса отвергли и распяли
	\item Бог воздвиг Иисуса из мёртвых
	\item в Иисусе Христе возможно прощение грехов, невозможное по Закону
\end{itemize}

%%св-во о ч-ке и мире
Апостол Павел свидетельствует, что человек может быть прощён и спасён, если примет Иисуса Христа.
Мир находится в решающей стадии: пророчества сбываются, Господь "дело делает" своё, и для каждого встаёт вопрос о покаянии.

%% призыв к Покаянию
Свой призыв апостол Павел произносит словами, "сказанными у Пророков": "Посмотрите, презрители, подивитесь и сгиньте, потому что дело делаю Я в дни ваши, дело, которому вы никак не поверите, если кто расскажет вам" (ср. Ис 5:24, 28:14-15)

По итогам проповеди, апостола со спутниками просили проповедовать в следующую субботу, а многие последовали за ними и начали научаться.
Слухи разошлись и в следующую субботу "почти весь город собрался слушать слово Божие".

\subsubsection*{Проповедь ап Павла перед обвинителями в Иерусалиме (Деян 22:1-24)}
%%внешний контекст
В Иерусалиме, в храме, Апостол Павел был обвинён, якобы он осквернил храм введением в него язычников.
Едва не растерзанный толпой, он был взят под стражу римским трибуном с воинами, но попросил о возможности принести перед обвинителями оправдательную речь.
Трибун дал такую возможность.

%%структура проповеди и внутр. логика
\begin{center}
	\begin{longtable}{ |c|c|p{0.70\textwidth}| } 
		\hline
		стих & смысл & содержание \\
		\hline\hline
		22:1 & обращение & – \\
		22:3-5 & свидетельство & жизнь ап Павла до обращения: наставленный в Законе и гонитель христиан \\ 
		22:6-21 & свидетельство & обращение, исцеление Павла и послушание от Бога идти к язычникам \\
		22:8 & керигма & Иисус Назорей – Господь \\
		22:16 & керигма & крестясь и "призвав имя Его" человек "смывает" грехи \\
		22:21 & керигма & Бог посылает на проповедь к язычникам \\
		\hline
	\end{longtable}
\end{center}

%%керигма
Апостол Павел говорит в своё оправдание и больше приносит личное свидетельство, прямо и честно рассказывает о себе, проводя параллели между собой и слушающими ("как все вы сегодня" Деян 22:3).
Но, так как собственную жизнь он строит на Божьем откровении, его свидетельство в себе косвенно заключает керигму (как бы к нему самому обращённую):
\begin{itemize}
	\item Иисус – Господь
	\item в Иисусе – Спасение и прощение грехов
	\item откровение Божье теперь и для язычников
\end{itemize} 

%%св-во о ч-ке и мире
Свидетельство о человеке прежде всего в универсализме Божьего Слова, теперь обращённого ко всем народам, а также в возможности прощения грехов в крещении и "призвании имени" Иисуса Христа.

%% призыв к Покаянию
Апостол Павел не успевает призвать слушающих к покаянию, но приносит собственное свидетельство покаяния, в т.ч. публично признавая, что "сочувствовал и стерёг одежды убивавших" Стефана.

\subsection*{Сравнительный анализ миссионерских проповедей иудеям и "богобоязненным"}

В проповеди иудеям значительную роль играет возвещение исполнения ветхозаветных пророчеств.
Как для самих апостолов, так и для их окружения, выросшего в среде, где Закон определял всю общественную жизнь, а Пророки возвещали основные духовные чаяния, свидетельство о Христе в этих текстах играло значительную роль.
Их отношение к тексту, при этом, было довольно свободным, в свете уже состоявшейся жизни, смерти и воскресения Христа.
Так, Данн прибегает к термину цитаты-"пешер", цитаты-толкования, возникавшей "когда сводили воедино данность текста и данность евангельской традиции" (\cite{@dunn.edinstvo} § 24).
Такие "новозаветные цитаты из Ветхого Завета суть интерпретации", а не "независимый авторитет" (там же).
Опираясь на это свидетельство и толкуя их в свете Евангелия, апостолы проповедуют исполнение обетованного Богом, исполнение пророчеств, продолжая проповедь Христа о "приблизившемся Царстве".


В 6 из 8 проповедей проповедник прямо ссылается на пророческие тексты, и говорит об их исполнении.
В Антиохии Писидийской ап Павел говорит о Христе как "сыне Давида" и ссылается на проповедь «современного» ему пророка Иоанна Крестителя, а перед толпой гонителей в Деян 22 самого себя, воспитанного в Законе "у ног Гамалиила", ставит свидетелем исполнения ветхозаветных обетований и надежд.
Эти толкования отличаются от толкований "книжников" и звучат "со властью", так как исповедуют как уже случившееся то, что в писаниях только ожидалось.


Стефан исповедует перед смертью, что "Небеса отверсты" и вспоминает видения апокалиптических глав пророка Даниила.
В Антиохии ап Павел говорит об исполнении "милостей, обещанных Давиду" (Деян 13:33) и о том, что ныне "Дело делаю Я (Господь)" (13:41).
В проповеди после сошествия Духа Святого, апостол Пётр толкует это явление словами пророка Иоиля (и, в свою очередь, толкует пророка этим исполнением у всех на глазах).
После исцеления хромого у ворот храма, апостол Пётр говорит, что нет нужды удивляться чудесам, так как наступило время чудес, которое возвещали Пророки.
Это удивительное единодушие апостольского толкования пророческих текстов подражает эсхатологичности проповеди Самого Иисуса.


Эсхатологическая весть для знавших Писание и живших в среде, полной апокалиптических настроений и проповедей (напр. проповедь Иоанна Крестителя), разворачивало людей к знакомым словам Писания, а слова – раскрывала по-новому для слушателей (подобно тому как делал Иисус, напр. в Лк 24:12-35).
Пророчество о Дне Господа, о наступлении Царства раскрывалось как уже сбывающееся.
Уже в проповеди Иоанна Крестителя зазвучала эта нота незамедлительности наступления Царства, и перед слушающим вставал вопрос: что в связи с этим нужно делать\footnote{Который они озвучивали прямо во время проповеди, судя по: Лк 3:10-14, Деян 2:37, 9:6, 16:30}?


На это у апостолов есть ответ, что также отличает их от иудейских апокалиптиков: "В то время как чаяния иудейской эсхатологии были неопределенными или выраженными чисто символическим языком, апокалиптическая надежда христианства сосредоточивалась на конкретном человеке" (\cite{@dunn.edinstvo}, § 69), Иисусе Христе.
Иисус исповедуется как Судья, Господь и Спаситель, а в Деян 3:13 – "отрок" Божий.
В Нём возвещается спасение Израиля и отпущение грехов, которые невозможно простить по Закону (Деян 13:38-39, 22:16).


Есть ещё один аспект проповеди иудеям со ссылкой на ветхозаветные писания.
Это обличение, что особенно раскрывается в толковании Стефаном священной истории как истории непослушания и отступничества, вплоть до нарушения воли Божьей строительством рукотворного храма.
У других проповедников обличение не столь радикально, но отговорка "отец у нас – Авраам" (Лк 3:8) обращена против иудеев.
Они не узнали Христа, прославленного Богом, Спасителя, и распяли его.
Потому все чаяния их веры теперь безнадёжны, если они не крестятся во имя Иисуса, ведь в Нём исполняется их обетование.


Наряду со смертью Иисуса, в 6 проповедях сказано о воскресении (в разговорах с Корнилием и евнухом Эфиопом этот эпизод не приведён).
Это – ещё одно прямое возвещение того, о чём в словах пророках только упомянается и что едва ли поддавалось внятному толкованию вне победы Иисуса Христа над смертью.


В эпизодах с "боящимися Бога" Эфиопом и Корнилием с домочадцами и друзьями есть так же возвещение универсальности Божьего призыва и крещения во имя Иисуса.
Вера – не для определённого народа или физической чистоты, но от чистоты сердца "боящихся Его (Бога) и делающих правду" (Деян 10:34-35).
Обретающие таким образом Христа как Господа получают прощение своих грехов и освобождение от них и дары Святого Духа (Деян 10:47).


%%% Local Variables:
%%% mode: latex
%%% TeX-master: "rpz"
%%% End:


