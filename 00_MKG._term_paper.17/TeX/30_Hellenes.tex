%&tex
\chapter{Миссиологический анализ проповедей иудеям}
\label{cha:hellenes}

В этом разделе рассмотрены 4 проповеди: в Листрах, в Афинах, перед прокуратором Феликсом, а также перед царём Агриппой \footnote{Хотя Агриппа II и был по происхождению иудей, он рос в Риме при императорском дворце и был скорее римского воспитания, чем иудейского. Его сестра, Друзилла, упомянутая в Деян 24, открыто перешла в язычество и была замужем за прокуратором Феликсом.}.

\section{Миссиологический анализ проповедей иудеям}
\subsubsection*{Проповедь ап Павла в Листрах (Деян 14:8-18)}
% контекст
После гонений со стороны иконийских иудеев, ап Павел со спутниками бежал в Ликаонские города.
В городе Листра ап Павел исцелил хромого от рождения человека, что вызвало своеобразное религиозное пробуждение в местных жителях.
Решив, что перед ними – ходящие по земле боги, они собрались принести им жертву.
Остановив их, ап Павел и Варнава обратились к ним с проповедью.

%%структура проповеди и внутр. логика
\begin{center}
	\begin{tabular}{ |c|c|c| }
		\hline
		стих & смысл & содержание \\
		\hline\hline
		14:15a & обращение, свидетельство & апостолы – не боги, а люди, несущие благую весть \\
		14:15b & керигма & человек должен обратиться от "суетных богов" к Богу живому, Творцу \\
		14:16-17 & керигма & Бог живой свидетельствовал о Себе людям добром и радостью, оставляя людям свободу \\
		14:18 & призыв & не приносить жертвы "суетным богам" \\
		\hline
	\end{tabular}
\end{center}

%%керигма
Керигма в проповеди ап Павла и Варнавы:
\begin{itemize}
	\item человек должен отказаться от идолов ради служения одному Богу живому
	\item истинный Бог – Творец мира
	\item от Бога – всё доброе и радость в жизни всех людей
	\item Бог оставляет человеку свободу "ходить своими путями"
	\item Бог "свидетельствует о Себе" и призывает к Себе
\end{itemize}

%%св-во о ч-ке и мире
Мир был сотворён Богом, и всё доброе в мире – от Бога. 
Идолы – "суетные боги", которым не следует поклоняться.
Человек свободен выбирать свой путь.

%% призыв к Покаянию
В словах проповеди, изложенных в Деян 14:8-18 не звучит прямого призыва, но из приведённых слов и фразы "успокоили народ, чтобы не приносили им жертвы", можно заключить наличие призыва оставить идолопоклонство.

%%реакция на проповедь
Несмотря на клевету на апостола и Варнаву со стороны иудеев, в Листре и близлежащих городах появились ученики и общины, которые они после наставляли (Деян 21-23).

\subsubsection*{Проповедь ап Павла в Ареопаге (Деян 17:16-34)}

% контекст
Апостол Павел ожидал спутников в Афинах и "дух в нём возмущался, видя, что город полон идолов".
Помимо обычной проповеди в синагоге, он начал разговаривать "на полащди каждый день со случайными встречными".
Философы и другие, кто "ничем другим не заполняли свои досуги как тем, чтобы говорить или слушать что-нибудь новое", заинтересовались и попросили рассказать в Ареопаге об этом "новом учении".

%%структура проповеди и внутр. логика
\begin{center}
	\begin{tabular}{ |c|c|c| }
		\hline
		стих & смысл & содержание \\
		\hline\hline
		17:22 & обращение & "по всему вижу, что вы особенно богобоязненны" \\
		17:23 & напоминание & жертвенник "неведомому богу" \\
		17:24 & керигма & Бог Творец и Господь неба и земли не живёт в рукотворённых храмах \\
		17:25 & керигма & Бог – источник жизни \\
		17:26 & керигма & Бог сотворил людей и поставил "сроки и пределы их обитанию" \\
		17:27a & керигма & призвание человека – искать Бога \\
		17:27b & керигма & Бог "не далеко от каждого из нас" \\
		17:28a & керигма & Бог – источник бытия и движения \\
		17:28b & напоминание & "как и некоторые из ваших поэтов сказали: «Ведь мы Его и род»" \\
		17:29 & керигма & Бог не подобен идолу, сделанному человеком \\
		17:30 & керигма, призыв & Бог "теперь" возвещает людям, "всем и всюду", покаяние \\
		17:31a & керигма & будет день Суда вселенной "по праведности" \\
		17:32b & керигма & Суд – через "поставленного" Им Мужа, воскрешённого из мёртвых \\
		\hline
	\end{tabular}
\end{center}


%%керигма
Керигма проповеди ап Павла:
\begin{itemize}
	\item единый истинный Бог – Творец и Господь неба и земли
	\item Бог – начало всего: жизни, бытия, движения
	\item Бог не подобен идолам и созданиям человеческой мысли и не нуждается в рукотворённых храмах
	\item человек сотворён Богом и призван Его искать
	\item Бог "не далеко" от каждого человека
	\item Бог лично призывает каждого человека покаяться
	\item грядёт Суд над вселенной "по праведности"
	\item Бог избрал Мужа, Который будет судить и Которого воскресил из мёртвых
\end{itemize}

%%св-во о ч-ке и мире
Апостол свидетельствует о сотворённости и конечности мира, а также о том, что Суд над миром будет "по праведности".
Человек сотворён Богом, Бог всегда рядом, человек призван Его найти и прийти к Нему, ответив на Его призыв покаянием.
В "Муже, Которого Он поставил" побеждена смерть.

%% призыв к Покаянию
Апостол объявляет, что Бог лично изменил ход истории: времена неведения прошли, и Бог теперь возвещает "всем и всюду" необходимость покаяния.
Покаяние нужно, так как Суд над миром – "по праведности".

%%реакция на проповедь
Слова о воскресении мёртвых вызывают непонимание и многие начинают насмехаться, хотя кое-кто просит послушать об этом ещё раз.
Некоторые из слушавших уверовали (Деян 17:34).

\subsubsection*{Проповедь ап Павла прокуратору Феликсу (Деян 24:10-26)}
% контекст
Обвиняемый первосвященником Ананией и другими старейшинами в осквернении храма, ап Павел предстал перед судом прокуратора Феликса.
Когда апостолу позволили говорить в свою защиту, он произнёс проповедь.

%%структура проповеди и внутр. логика
\begin{center}
	\begin{tabular}{ |c|c|c| }
		\hline
		стих & смысл & содержание \\
		\hline\hline
		24:10 & обращение & – \\
		24:11-13 & апология & Павел не совершил ничего дурного \\
		24:14-15 & свидетельство & "служу Богу отцов наших... надежду имея на Бога" \\ 
		24:14 & керигма & есть Путь служения Богу, согласно с Законом и Пророками \\
		24:15 & керигма & будет воскресение праведных и неправедных \\
		24:16 & свидетельство & потому апостол старается жить в непорочной совести \\
		24:17-20 & апология & Павел не совершил ничего дурного \\
		24:21 & свидетельство & ап Павел верует в воскресение мёртвых и готов на том стоять \\
		24:25 & призыв & призыв к "праведности" и "обладанию собой", т.к. будет Суд \\
		\hline
	\end{tabular}
\end{center}

%%керигма
Керигма в проповеди ап Павла:
\begin{itemize}
	\item существует Путь служения Богу
	\item грядёт воскресение праведных и неправедных
	\item требуется жить в "непорочной совести пред Богом и людьми", праведно, обладая собой
\end{itemize}


%%св-во о ч-ке и мире
Апостол свидетельствует, что всякого человека ожидает воскресение, и от всякого требуется праведность и чистота совести.
Что же касается воскресения Христа, то до личной встречи с Феликсом и Друзиллой (24:24-26), апостол не говорит об этом.
Возможно, причина в том, чтобы проповедь изабвления от смерти не оказалась "смазана" оттенком внутренних распрей иудеев о неких спорных духовных авторитетах.

%% призыв к Покаянию
После 24:21 Феликс, "имея более точные сведения о Пути", отослал первосвященника и обвинителей и поместил ап Павла под стражу, чтобы через несколько дней прийти к нему с женой Друзиллой, Иудеянкой и "слушать его о вере во Христа Иисуса".
Тогда ап Павел продолжил проповедь призывом к покаянию и праведной жизни.
Из Деян 24:26 следует, что безуспешно, так как прокуратор Феликс испугался слов о будущем суде и "надеялся, что Павел даст ему денег".


\subsubsection*{Проповедь ап Павла перед царём Агриппой (Деян 26:1-31)}
% контекст
Не желая, чтобы его отдали в руки иудеев, способных убить его, ап Павел потребовал суда Кесаря (императора), что позволялось римскому гражданину под угрозой смертного приговора.
По приезду царя Агриппы, апостолу было позволено держать защитительную речь перед ним, "трибунами и виднейшими в городе людьми", и он произнёс проповедь. 

%%структура проповеди и внутр. логика
\begin{center}
	\begin{tabular}{ |c|c|c| }
		\hline
		стих & смысл & содержание \\
		\hline\hline
		26:2-3 & обращение & царь Агриппа назван "знатоком всех обычаев и спорных мнений Иудеев" \\
		26:4-22 & свидетельство & история воспитания, обращения, исцеления и служения апостола \\
		26:6-8 & напоминание & Иудеи, "усердно служа Богу", надеются на обещание от Бога \\
		26:15 & керигма & Иисус – Господь \\
		26:17 & керигма & Иисус имеет слово для язычников \\
		26:18 & керигма, призыв & необходимо обратиться "от тьмы к свету и от власти сатаны к Богу" \\
		26:18b & керигма & отпущение грехов возможно в вере в Иисуса \\
		26:20 & призыв & апостол возвещал язычникам каяться и творить "достойные покаяния" дела \\
		26:23 & керигма & Христос пострадал, но воскрес из мёртвых \\
		26:23b & керигма & проповедь Христа – для всех народов \\
		26:25 & апология & проповедь апостола – "слова истины и здравого смысла" \\
		26:26 & призыв & если Агриппа верит Пророкам, ему надо поверить в проповедь ап Павла \\
		26:29 & свидетельство & апостол надеется, что все уверуют во Христа \\
		\hline
	\end{tabular}
\end{center}

%%керигма
Керигма в проповеди ап Павла:
\begin{itemize}
	\item Иисус – Господь
	\item проповедь Христа – для всех народов
	\item во Христе возможно отпущение грехов
	\item Христос пострадал, но воскрес из мёртвых
	\item необходимо освобождение от власти сатаны и обращение к Богу, покаяние и творение "достойных покаяния дел"
\end{itemize}

%%св-во о ч-ке и мире
Апостол свидетельствует, что человек призван к свободе от власти греха и сатаны.
Проповедь Христа возвещает всем людям освобождение и отпущение грехов в вере в Него.
Обетование и надежда Иудеев – на воскресение, и это сбывается во Христе.

%% призыв к Покаянию
Апостол прямо обращается к царю Агриппе с призывом откликнуться на его проповедь, если он верит словам Пророков.
Как надлежит откликнуться – он говорит ранее, рассказывая о язычниках, которые покаялись и теперь творят "достояные покаяния дела".

%%реакция на проповедь
Неизвестно, какими были плоды проповеди кроме того, что Агриппа и его приближённые нашли, "что ничего достойного смерти или уз человек этот не делает".


\section{Особенности миссионерской проповеди язычникам в Новом Завете}
Основной особенностью проповеди язычникам стала необходимость проповедовать Единого Бога Творца и призывать к отказу от идолопоклонства.
В двух проповедях (Листры и Афины) присутствует это свидетельство, разное по форме, но единое в содержании.
Во всех случаях Бог исповедуется как:  
\begin{itemize}
	\item Творец и начало всего
	\item единый Бог для всех, а не для какого-либо народа
	\item Живой Бог, лично принимающий участие в жизни человека, с Которым возможны личные отношения
	\item истинный, отличающийся от созданий человеческих рук и мыслей (идолов)
	\item имеющий благое попечение о человеке и человечестве
\end{itemize}

В разговоре с простыми людьми (судя по всему, Листра – небольшой город), апостол ссылается не на философов и поэтов, а на действие Божье в жизни его слушателей: "подавая вам с неба дожди и времена плодоносные, исполняя пищею и радостью сердца ваши". 
Бог как пища, радость, дождь – един для всех и для каждого, близок и благ к каждому.
Кроме того, Он "позволил всем народам ходить своими путями", а значит лично принимает свободу человека и ждёт таких же личных отношений от него.


В Листрах Бог возвещается как создатель "неба и земли, и моря и всего, что в них", – как практически детским перечислением (ср. Пс 103) творения исповедуется Бог, как начало всего.
Совсем по-другому – в Ареопаге: "Сам дарует всем жизнь и дыхание и всё" (начало жизни), "произвёл род человеческий... предуставив сроки и пределы их обитанию", "«Ведь мы Его и род»" (начало людей), "в Нём мы живём и движемся и существуем" (начало бытия и движения).
Бог – конечный ответ на поиск "начал" в философии и личности человека (Отец, хотя бы как основатель рода человеческого).
И во всём – с акцентом на личностности Бога: "Бог не далеко от каждого из нас", "произвёл... род человеческий... искать Бога", – т.е., так же оставил свободу выбора пути, хоть и с личной заинтересованностью в близости.


Таким образом, в откровении о Едином Боге говорится и откровение о человеке как свободном, но призванном к духовному пути близости к своему Творцу.
Но человек ещё только призван стать таким, обратиться "от тьмы к свету и от власти сатаны к Богу" (Деян 26:18), получить отпущение грехов.
Как и в Лк, в Деян Царство Божье проповедуется как избавление человека от власти духов мира сего, от власти болезней и немощей и, в итоге, смерти (чему соответствует проповедь воскресения Христова как исполнения обетованного Пророками).


Эта свобода актуализируется в оставлении служения "руками человеческими" созданиям "печати искусства и мысли человеческой" в "рукотворённых храмах".
В проповедях перед Ареопагом, перед прокуратором Феликсом и перед царём Агриппой апостол наряду с возвещением свободы говорит о суде и покаянии, которые отличаются от привычных методов оценки человеческой жизни божественными силами (либо равнодушными, либо смотрящими на богатство приносимых даров).
Суд над вселенной будет "по праведности" (Деян 17:31), язычники, которым проповедовал ап Павел "обращались к Богу, творя дела, достойные покаяния" (Деян 26:18), и с прокуратором Феликсом апостол начинает говорить о необходимости изменения жизни: "обладании собой", "праведности" (Деян 24:24-25).
В Листрах же ап Павел и Варнава буквально силой останавливают языческую церемонию, которой хотели отметить местные жители чудо исцеления хромого.
Бог проповедуется как имеющий всё и, соответственно, не "имеющий в чём-либо нужду" от человека, но призывающий человека к себе через покаяние (обращение) (Деян 17:30).


Этот призыв, как исповедует апостол, теперь обращён ко "всем и всюду", потому что проходят "времена неведения", и определён день суда (Деян 17:30-31). 
Эсхатологический элемент керигмы, пробуждающий актуальную память знавшего слова пророков, для язычников был приниципиально новым свидетельством о мире, времени, истории\footnote{См. напр. \cite{@dunn.vvedenie}: "Апокалиптическая эсхатология принимает еврейское видение истории (практически уникальное для античности): история мыслится линейно... как движение вперёд к определённому концу и цели" (§ 66.3)}.
Без этого элемента возвещения Суда и призыв к покаянию как возможности для Спасения, были бы невозможны.


Интересно, что проповедуя в Ареопаге, апостол Павел отталкивается от архаического (вероятно, даже в то время) образа-мифа "неведомого бога", а не от представления о мире одной из популярных в то время философских школ.
Судя по цитатам, он был знаком с культурой и взглядами на устройство мира эллинистического мира.
Но, обращается к глубоко религиозному чувству трепета перед неведомым, тем отметая возможность расценить его речь как приглашение к философскому дискурсу.
Эта проповедь "не в убедительных словах человеческой мудрости" (1Кор 2:4) вызывает непонимание одних ("одни насмехались"), но интерес других ("мы послушаем тебя об этом ещё раз").


В рамках того же религиозного трепета перед неведомым, ап Павел в проповеди язычникам аппелирует к столь же неразрешимым загадкам сил природы и смерти.
Если по первому вопросу в иудейском мире ещё существовал ответ: надо поклоняться не силам, а Богу, от Которого всё произошло, то последний ставил в тупик и иудеев, и эллинов.
Его разрешение было возможно только в керигме Господа над всем, включая смерть, от которой Его избавляет Бог.


Апостол проповедовал то, что, вероятно, когда-то "пронзило" его самого, ведь "для самого апостола Павла важнейшим эсхатологическим событием, в котором живой Бог открыл... Свой замысел о спасении мироздания, стало \textit{воскресение Иисуса из мёртвых}"\cite{@wright.4to}.
Победа над смертью означала победу над грехом, а значит – полное освобождение от власти сатаны, полный и окончательный переход под власть Бога.
Отсюда – постоянство керигмы о воскресении Христа, несмотря на неготовность слушателей эллинистического воспитания.


%%% Local Variables:
%%% mode: latex
%%% TeX-master: "rpz"
%%% End:

