\chapter{Проповеди иудеям}
\label{cha:appendix1}

\subsection*{Проповедь ап Петра в день Пятидесятницы (Дян 2:14-2:47)}
14 Петр же, став с одиннадцатью, возвысил свой голос и возгласил им: мужи Иудейские и все, живущие в Иерусалиме! Да будет вам это известно и вслушайтесь в слова мои.

15 Ибо не пьяны они, как вы предполагаете: ведь только третий час дня.

16 Но это то, что сказано чрез пророка Иоиля:

17 "И будет в последние дни, говорит Бог: изолью от Духа Моего на всякую плоть, и будут пророчествовать сыны ваши и дочери ваши, и юноши ваши будут видеть видения. И старцам вашим будут сниться сны,

18 и на рабов Моих и на рабынь Моих в те дни изолью от Духа Моего, и будут пророчествовать.

19 И дам чудеса на небе вверху и знамения на земле внизу, кровь и огонь и клубы дыма.

20 Солнце превратится в тьму, и луна в кровь, прежде чем придет день Господень великий и блистающий.

21 И будет: всякий, кто призовет имя Господне, будет спасен".

22 Мужи Израильские! Выслушайте эти слова: Иисуса Назорея, Мужа, Богом отмеченного для вас силами, и чудесами, и знамениями, которые сотворил чрез Него Бог среди вас, как вы сами знаете,

23 Его, по определению и предведению Божию преданного, вы, пригвоздив рукою беззаконных, убили.

24 Бог Его воскресил, разрешив муки смерти, потому что Он не мог быть держим ею.

25 Ибо Давид говорит о Нем: "Я видел Господа предо мною непрестанно, ибо Он по правую мою сторону, дабы я не поколебался.

26 Оттого возрадовалось мое сердце и возликовал язык мой, даже и плоть моя пребудет в надежде,

27 ибо Ты не оставишь души моей во аде и не дашь святому Твоему увидеть тление.

28 Ты поведал мне пути жизни, Ты исполнишь меня радости пред лицом Твоим".

29 Мужи братья! Да позволено будет сказать вам с дерзновением о патриархе Давиде, что он и скончался и погребен, и гробница его у нас до сего дня.

30 Итак, будучи пророком и зная, что клятвою поклялся ему Бог от плода чресл его посадить на престоле его,

31 - он, провидя, изрек о воскресении Христа, что и Он не оставлен во аде, и плоть Его не увидела тления.

32 Этого Иисуса воскресил Бог, чему все мы свидетели.

33 Итак, десницей Божией вознесённый, и приняв от Отца обещанного Духа Святого, Он излил то, что вы и видите и слышите.

34 Ибо Давид не восшел на небеса, но он сам говорит: "Сказал Господь Господу Моему: сядь по правую сторону Мою,

35 доколе Я не положу врагов Твоих в подножие ног Твоих".

36 Итак, твердо знай весь дом Израилев, что и Господом и Христом сделал Его Бог, Того Иисуса, Которого вы распяли.

37 Они же, услышав, поражены были в самое сердце, и сказали Петру и остальным апостолам: что нам делать, мужи братья?

38 А Петр им сказал: покайтесь, и да крестится каждый из вас во имя Иисуса Христа для отпущения грехов ваших, и вы получите дар Святого Духа.

39 Ибо для вас обещание и для детей ваших и для всех дальних, кого ни призовет Господь Бог наш.

40 И иными многими словами он свидетельствовал и увещал их, говоря: спасайтесь от этого извращенного рода.

41 Итак, приняв слово его, они были крещены. И присоединилось в день тот душ около трех тысяч.

42 И пребывали они постоянно в учении апостолов и в общении, в преломлении хлеба и в молитвах.

43 Был же во всякой душе страх. И много чудес и знамений совершалось чрез апостолов.

44 А все уверовавшие были вместе и всё у них было общее.

45 И то, чем владели и что имели, они продавали и разделяли это всем, смотря по нужде каждого.

46 И каждый день единодушно пребывая в храме, и преломляя по домам хлеб, они принимали пищу в веселии и простоте сердца,

47 хваля Бога и будучи в милости у всего народа. А Господь прибавлял к ним спасаемых каждый день.

\subsection*{Проповедь апп Петра и Иоанна перед первосвященниками (Деян 4:5-21)}

5 И было на следующий день, что собрались в Иерусалиме начальники их и старейшины и книжники,

6 и Анна первосвященник и Каиафа, и Иоанн и Александр, и все, кто были из первосвященнического рода,

7 и, поставив их посредине, допрашивали: какой силой или каким именем вы это сделали?

8 Тогда Петр, исполнившись Духа Святого, сказал им: начальники народа и старейшины,

9 если нам сегодня чинят допрос о благодеянии человеку больному, каким образом он спасен,

10 то да будет известно вам и всему народу Израильскому, что именем Иисуса Христа Назорея, Которого вы распяли, Которого Бог воздвиг из мертвых, – Им стоит он перед вами здоровый.

11 Он есть Камень, признанный негодным вами – строителями, оказавшийся во главе угла.

12 И нет ни в ком другом спасения. Ибо под небом нет и иного имени, данного людям, которым надлежит нам быть спасенными.

13 Видя же дерзновение Петра и Иоанна, и поняв, что они люди некнижные и простецы, удивлялись и признавали их, что они были с Иисусом.

14 И видя, что человек, исцеленный, с ними стоит, ничего не могли возразить.

15 И, приказав им выйти вон из синедриона, совещались друг с другом

16 и говорили: что нам делать с этими людьми? Ведь то, что ими совершено замечательное знамение, это явно всем живущим в Иерусалиме, и мы не можем отрицать;

17 но, чтобы оно еще больше не распространилось в народе, пригрозим им, чтобы они уже от этого имени не говорили никому из людей.

18 И, призвав их, приказали совсем не говорить и не учить во имя Иисуса.

19 Но Петр и Иоанн сказали им в ответ: рассудите, справедливо ли пред Богом, слушать вас больше чем Бога?

20 Ибо не можем мы не говорить о том, что видели и слышали.

21 И, снова пригрозив, отпустили их, так и не находя повода наказать их, из-за народа, потому что все прославляли Бога за происшедшее.

\subsubsection*{Продолжение проповеди (Деян 5:27-33)}
27 И приведя их, поставили в синедрионе. И спросил их первосвященник:

28 мы строжайше приказали вам не учить от этого имени. И вот, вы наполнили Иерусалим учением вашим и хотите навести на нас кровь Этого Человека.

29 Но Петр и апостолы сказали: повиноваться должно Богу больше чем людям,

30 Бог отцов наших воздвиг Иисуса, Которого вы умертвили, повесив на древе:

31 Его Бог вознес десницей Своей, как Начальника и Спасителя, чтобы дать Израилю покаяние и отпущение грехов.

32 И мы свидетели тому, и Дух Святой, Которого Бог дал повинующимся Ему.

33 Слышавшие были в бешенстве и хотели убить их.

\subsection*{Проповедь Стефана (Деян 7:2-60)}
2 И тот сказал: Мужи братья и отцы! Послушайте! Бог славы явился отцу нашему Аврааму, когда он был еще в Месопотамии, прежде поселения его в Харране, и сказал ему:

3 "выйди из земли твоей и из родни твоей и иди в землю, которую Я тебе укажу".

4 Тогда, выйдя из земли Халдейской, поселился он в Харране. И оттуда, по смерти отца его, Бог переселил его в эту землю, на которой вы теперь живете.

5 И не дал ему наследия в ней ни на стопу ноги, но обещал дать ее во владение ему и семени его после него, обещал, когда не было у него детей.

6 И Бог сказал так: "будет семя его пришельцем на земле чужой, и поработят его и будут творить ему зло четыреста лет.

7 И народ, которому они будут рабами, судить буду Я" - Бог сказал, - "и после этого они выйдут и будут служить Мне на этом месте".

8 И дал ему завет обрезания. И так родил он Исаака и обрезал его в день восьмой; и Исаак – Иакова, и Иаков – двенадцать патриархов.

9 И патриархи, позавидовав Иосифу, продали его в Египет. И был Бог с ним,

10 и избавил его от всех скорбей и дал ему милость и мудрость в глазах фараона, царя Египетского. И поставил его начальником над Египтом и над всем домом его.

11 И пришел голод на весь Египет и Ханаан, и скорбь великая; и не находили пищи отцы наши.

12 Иаков же, услышав, что есть хлеб в Египте, послал отцов наших в первый раз.

13 И во второй раз узнан был Иосиф братьями своими, и стал известен фараону род Иосифа.

14 Иосиф же послал и вызвал Иакова, отца своего, и всю родню: семьдесят пять душ.

15 И сошел Иаков в Египет и скончался сам и отцы наши;

16 и были они перенесены в Сихем и положены в гробнице, которую купил Авраам ценой серебра у сынов Емморовых в Сихеме.

17 Когда же стало приближаться время исполниться обещанию), которое Богу угодно было дать Аврааму, возрос народ и умножился в Египте.

18 Так было, пока не восстал иной царь над Египтом, который не знал Иосифа.

19 Он, прибегнув к хитрости против рода нашего, в злобе на отцов, заставлял их бросать младенцев своих, чтобы те не оставались в живых.

20 В это время родился Моисей и был прекрасен пред Богом. Его кормили три месяца в доме отца.

21 А когда был он брошен, взяла его к себе дочь фараона и воспитала его как своего сына.

22 И научен был Моисей всей мудрости Египетской и был силен в словах и делах своих.

23 Когда же исполнялось ему сорок лет, пришло ему на сердце посетить братьев своих, сынов Израилевых.

24 И увидев, что кого-то обижают, он заступился и отомстил за страдавшего, поразив Египтянина.

25 Он думал, что братья понимают, что Бог рукой его дает спасение им; но они не поняли.

26 На следующий день он появился между ними, когда они дрались, и склонял их к миру словами: "мужи, вы – братья; почему вы обижаете друг друга?"

27 Но обижающий ближнего оттолкнул его, сказав: "кто тебя поставил начальником и судьей над нами?

28 Не хочешь ли ты убить меня так же, как убил вчера Египтянина?"

29 И бежал Моисей из-за этого слова и стал пришельцем в земле Мадиамской, где он родил двух сыновей,

30 И по исполнении сорока лет, явился ему в пустыне горы Синая Ангел в пламени горящей купины.

31 Моисей же, увидев, дивился видению. А когда он подходил, чтобы всмотреться, раздался голос Господа:

32 "Я – Бог отцов твоих, Бог Авраама, и Исаака, и Иакова". Объятый трепетом, Моисей не посмел всматриваться,

33 И сказал ему Господь: "сними обувь с ног твоих, ибо место, на котором ты стоишь, земля святая.

34 Увидел Я, увидел притеснение народа Моего, который в Египте, и стенание его услышал и нисшел освободить их. И теперь иди, Я пошлю тебя в Египет".

35 Этого Моисея, которого они отвергли, сказав: "кто тебя поставил начальником и судьей?" - его-то послал Бог и начальником и избавителем, рукою Ангела, явившегося ему в купине.

36 Это он вывел их, творя чудеса и знамения в земле Египетской и в Чермном море и в пустыне сорок лет.

37 Это тот Моисей, который сказал сынам Израилевым: "Пророка вам воздвигнет Бог из братьев ваших, как меня".

38 Это тот, кто в собрании в пустыне был с Ангелом, говорившим ему на горе Синае, и с отцами нашими; тот, который принял слова живые, чтобы дать вам,

39 которому не захотели покориться отцы ваши, но отринули его и обратились сердцами своими к Египту,

40 сказав Аарону: "сделай нам богов, которые пойдут впереди нас, ибо Моисей этот, который вывел нас из земли Египетской, – не знаем, что случилось с ним".

41 И сделали они тельца в дни те, и принесли жертву идолу, и радовались делу рук своих.

42 И отвратился Бог и предал их служить воинству небесному, как написано в книге Пророков: "Заклания и жертвы принесли ли вы Мне за сорок лет в пустыне, дом Израилев?

43 И взяли вы с собою скинию Молоха и звезду бога вашего Ремфана, изображения, которые вы сделали, чтобы поклоняться им, и Я переселю вас дальше Вавилона".

44 Скиния свидетельства была у отцов наших в пустыне, как повелел Говоривший Моисею сделать её по образу, который тот увидел.

45 Её, получив по преемству, и внесли отцы наши с Иисусом во владения народов, которых изгнал Бог от лица отцов наших. Так было до дней Давида,

46 который обрел благодать пред Богом и молился о том, чтобы найти жилище Богу Иакова.

47 Соломон же воздвиг Ему дом.

48 Но не живет Всевышний в рукотворенном, как говорит пророк:

49 "Небо – Мне престол, а земля – подножие ног Моих. Какой дом вы построите Мне, говорит Господь, или какое место покоя Моего?

50 Не Моя ли рука сотворила это всё?"

51 Жестоковыйные и необрезанные сердцем и ушами, вы всегда Духу Святому противитесь, как отцы ваши, так и вы.

52 Кого из пророков не гнали отцы ваши? И убили они предвозвестивших пришествие Праведного, Которого предателями и убийцами вы теперь сделались,

53 вы, которые получили Закон в наставлениях ангельских и не сохранили.

54 Слыша это, они кипели бешенством в сердцах своих и скрежетали на него зубами.

55 А он, полный Духа Святого, устремил взор к небу и увидел славу Божию и Иисуса, стоящего по правую сторону Бога,

56 и сказал: вот я вижу отверстые небеса и Сына Человеческого стоящего по правую сторону Бога.

57 Они же, закричав громким голосом, зажали уши свои и устремились единодушно на него;

58 и, выгнав его за пределы города, побивали камнями. И свидетели сложили одежды свои у ног юноши, называемого Савлом.

59 И побивали камнями Стефана, а он взывал и говорил: Господи Иисусе, прими дух мой.

60 И склонив колени, воскликнул громким голосом: Господи, не вмени им этого греха! И сказав это, почил.

\subsection*{Проповедь ап Филиппа эфиопскому евнуху (Деян 8:26-40)}
26 А ангел Господень сказал Филиппу: встань и к полудню иди на дорогу, спускающуюся от Иерусалима к Газе. Она безлюдна.

27 И встав, он пошел. И вот муж Эфиоп, евнух, вельможа Кандакии, царицы Эфиопской, который был над всей ее казной. Он ходил для поклонения в Иерусалим,

28 и возвращался. И сидел он на своей колеснице и читал пророка Исаию.

29 И Дух сказал Филиппу: подойди и пристань к этой колеснице,

30 И подбежав, Филипп услышал, что он читает Исаию пророка,

31 и сказал: да понимаешь ли ты, что читаешь? А он сказал: как мне понять, если кто не наставит меня? И попросил Филиппа подняться и сесть с ним.

32 А место Писания, которое он читал, было такое: Как овца на заклание Он был приведен, и как агнец пред стригущим Его безгласен, так Он не отверзает уст Своих.

33 В унижении Его, было отказано Ему в правосудии. Род Его кто изъяснит? Ибо жизнь Его изъемлется от земли.

34 И обращаясь к Филиппу, евнух сказал: прошу тебя, о ком пророк говорит это: о себе или о ком-нибудь другом?

35 Филипп же отверз уста свои и, начав от этого писания, благовествовал ему Иисуса.

36 Двигались они по дороге и пришли к какой-то воде. И говорит евнух: вот вода; что препятствует мне креститься?

37 И сказал ему Филипп: если веруешь от всего сердца, можно. Он же сказал в ответ: верую, что Иисус Христос есть Сын Божий.

38 И приказал остановить колесницу, и сошли оба в воду, как Филипп, так и евнух, и он крестил его.

39 Когда же они вышли из воды, Дух Господень восхитил Филиппа, и больше не видел его евнух, и продолжал он, радуясь, свой путь.

40 А Филипп оказался в Азоте и, проходя, благовествовал всем городам, доколе не пришел в Кесарию.

\subsection*{Проповедь ап Петра сотнику Корнилию с домочадцами и друзьями (Деян 10:24-48)}
24 А на следующий день вошли они в Кесарию. Корнилий же ожидал их, созвав родственников своих и ближайших друзей.

25 А когда Петр готов был войти, Корнилий, встретив его, пал к его ногам и поклонился.

26 Но Петр поднял его говоря: встань, сам я тоже человек.

27 И беседуя с ним, вошел, и находит многих в сборе,

28 и сказал он им: вы знаете, как незаконно для Иудея сближаться с иноплеменником или приходить к нему; а мне Бог указал не называть ни одного человека скверным или нечистым.

29 Поэтому я и пришел беспрекословно, будучи позван. Итак, я спрашиваю: по какой причине вы послали за мной?

30 И Корнилий сказал: в этот час пошел четвертый день, как я, в девятый час, молился у себя в доме, и вот стал передо мною муж в одежде блестящей

31 и говорит: Корнилий, услышана твоя молитва, и милостыни твои воспомянуты пред Богом.

32 Итак, пошли в Иоппию за Симоном, который прозывается Петром: он гостит в доме Симона Кожевника близ моря".

33 Поэтому я тотчас же послал к тебе, и ты хорошо сделал, что пришел. Теперь же мы все пред Богом, чтобы выслушать всё, что повелено тебе Господом.

34 Петр отверз уста и сказал: Поистине я убеждаюсь, что Бог нелицеприятен,

35 но во всяком народе боящийся Его и делающий правду приятен Ему.

36 Он послал сынам Израилевым слово, благовествуя мир чрез Иисуса Христа: Этот есть Господь всех.

37 Вы знаете, что произошло по всей Иудее, начиная от Галилеи, после крещения, которое проповедал Иоанн:

38 об Иисусе из Назарета, как помазал Его Бог Духом Святым и силою, и Он ходил, благотворя и исцеляя всех угнетаемых диаволом, потому что Бог был с Ним.

39 И мы свидетели всего, что сделал Он в стране Иудейской и в Иерусалиме. Его они и убили, повесив на древе.

40 Его Бог воздвиг в третий день и дал Ему являться

41 – не всему народу, но свидетелям, предызбранным Богом, нам, которые с Ним ели и пили по воскресении Его из мертвых.

42 И Он повелел нам проповедать народу и засвидетельствовать, что Он есть поставленный Богом Судия живых и мертвых.

43 О Нем все пророки свидетельствуют, что всякий верующий в Него получит отпущение грехов именем Его.

44 Петр еще произносил эти слова, как сошел Дух Святой на всех слышащих слово.

45 И изумились верующие из обрезанных, которые пришли с Петром, что дар Святого Духа излился и на язычников;

46 ибо они слышали их, говорящих языками и величающих Бога.

47 Тогда ответил Петр: может ли кто отказать в воде крещения тем, кто приняли Духа Святого, как и мы?

48 И велел им креститься во имя Иисуса Христа. Тогда они попросили его остаться на несколько дней.

\subsection*{Проповедь ап Павла в Антиохии Писидийской (Деян 13:14-44)}
14 Они же, пройдя путь от Пергии, прибыли в Антиохию Писидийскую и, придя в синагогу в день субботний, сели.

15 И после чтения Закона и Пророков начальники синагоги послали им сказать: мужи братья, если у вас есть слово наставления к народу, говорите.

16 И Павел, встав и дав знак рукой, сказал: мужи Израильские и боящиеся Бога, послушайте.

17 Бог народа этого, Израиля, избрал отцов наших и возвысил народ во время пребывания в земле Египетской, и рукою высокою вывел их из нее,

18 и около сорока лет терпел их в пустыне

19 и, истребив семь народов в земле Ханаанской, дал им в наследие землю их

20 приблизительно на четыреста пятьдесят лет. И после этого дал судей до Самуила пророка.

21 И затем они просили царя, и дал им Бог Саула, сына Киса, мужа из колена Вениаминова, на сорок лет.

22 И отстранив его, Он воздвиг им в цари Давида, о котором и сказал во свидетельство: "Я нашел Давида, сына Иессея, мужа по сердцу Моему, который исполнит всю волю Мою".

23 От его-то семени Бог, по обещанию, привел Израилю Спасителя Иисуса

24 после того, как Иоанн проповедовал, перед самым явлением Его, крещение покаяния всему народу Израильскому.

25 А когда Иоанн завершил свое поприще, он говорил: "я не то, что вы обо мне думаете; но вот, идет за мною Тот, у Которого я недостоин развязать обувь на ногах".

26 Мужи братья, сыны рода Авраамова, и боящиеся Бога между вами: нам послано слово спасения этого.

27 Ибо живущие в Иерусалиме и начальники их, не узнав Его, исполнили и голоса пророков, читаемые каждую субботу, осудив Его.

28 И не найдя никакого основания для смерти, упросили Пилата убить Его.

29 Когда же исполнили всё написанное о Нем, то, сняв с древа, положили Его в гробницу,

30 Но Бог воздвиг Его из мертвых.

31 Он являлся в течение многих дней пришедшим вместе с Ним из Галилеи в Иерусалим. Они теперь свидетели Его перед народом.

32 И мы вам благовествуем обещание, которое дано было отцам,

33 что Бог его исполнил для нас, их детей, воскресив Иисуса, как и написано в псалме втором: "Ты – Сын Мой, Я сегодня родил Тебя".

34 А что Он воскресил Его из мертвых, так что Он уже не обратится в тление, – Он сказал так: "Я дам вам святое достояние Давидово непреложное".

35 Поэтому Он и в другом месте говорит: "Ты не дашь святому Твоему увидеть тление".

36 Но Давид, послужив в своем поколении совету Божию, почил и приложился к отцам своим и увидел тление.

37 Тот же, Кого Бог воздвиг, не увидел тления.

38 Итак, да будет известно вам, мужи братья, что ради Него вам возвещается отпущение грехов, и во всём, в чем вы не могли быть оправданы Законом Моисеевым,

39 всякий верующий оправдывается Им.

40 Итак, берегитесь, чтобы не пришло на вас сказанное у Пророков:

41 "Посмотрите, презрители, подивитесь и сгиньте, потому что дело делаю Я в дни ваши, дело, которому вы никак не поверите, если кто расскажет вам".

42 И когда они выходили, их просили, чтобы в следующую субботу им сказаны были эти слова.

43 И когда собрание было распущено, последовали многие из Иудеев и благоговейных прозелитов за Павлом и Варнавой, которые, беседуя с ними, убеждали их пребывать в благодати Божией.

44 А в следующую субботу почти весь город собрался слушать слово Божие.

\subsection*{Проповедь ап Павла перед обвинителями в Иерусалиме (Деян 22:1-24)}
1 Мужи братья и отцы, выслушайте мою нынешнюю перед вами защиту.

2 Услышав, что он начал свое обращение к ним на еврейском языке, они стали еще тише.

3 И он сказал: я – Иудей, рожденный в Тарсе Киликийском, но воспитанный в этом городе, у ног Гамалиила, наставленный во всей точности отеческого Закона, ревнитель по Боге, как все вы сегодня.

4 На этот Путь я воздвигнул смертельное гонение, заключая в узы и предавая в тюрьмы как мужчин, так и женщин,

5 в чем и первосвященник мне свидетель и весь совет старейшин; получив от них и письма к братьям, я шел в Дамаск с тем, чтобы и там находящихся привести в узах в Иерусалим для наказания.

6 И было со мной, когда я шел и приближался к Дамаску: около полудня внезапно воссиял с неба сильный свет вокруг меня,

7 и я упал на землю и услышал голос, говорящий мне: "Саул, Саул, что ты Меня гонишь?"

8 И я ответил: "кто Ты, Господи?" И Он сказал мне: "Я – Иисус Назорей, Которого ты гонишь".

9 Бывшие со мной свет видели, но голоса Того, Кто мне говорил, не слышали.

10 И я сказал: "что мне делать, Господи?" Господь же сказал мне: "встань и иди в Дамаск и там тебе будет сказано о всём, что назначено тебе делать".

11 И пока я ничего не видел от славы света того, бывшие со мной за руку привели меня в Дамаск.

12 А некий Анания, муж благочестивый по Закону, имеющий доброе свидетельство от всех местных Иудеев,

13 пришел ко мне и, подойдя, сказал мне: "Саул, брат, прозри!" И я в тот же час прозрел и устремил взор на него.

14 А он сказал: "Бог отцов наших предназначил тебя познать волю Его и увидеть Праведного и услышать голос из уст Его,

15 потому что ты будешь свидетелем Ему пред всеми людьми о том, что ты видел и слышал.

16 И теперь, что ты медлишь? Восстав, крестись и смой грехи твои, призвав имя Его".

17 И было со мной по возвращении в Иерусалим, когда я молился в храме: пришел я в исступление

18 и увидел Его, и Он говорил мне: "поспеши и выйди скорее из Иерусалима, потому что не примут твоего свидетельства о Мне".

19 сказал: "Господи, они знают, что я заключал в тюрьмы и бил в синагогах верующих в Тебя;

20 и когда проливалась кровь Стефана, свидетеля Твоего, я и сам стоял тут же и сочувствовал и стерег одежды убивавших его".

21 И Он сказал мне: "иди, потому что Я пошлю тебя далеко к язычникам".

22 До этого слова они его слушали, а тогда возвысили голос свой, говоря: долой с земли такого! Нельзя ему жить!

23 И пока они кричали и потрясали одеждами и бросали пыль в воздух,

24 трибун велел ввести его в казарму, сказав подвергнуть его допросу под бичем, чтобы узнать, по какой причине они так кричали на него.


%%% Local Variables: 
%%% mode: latex
%%% TeX-master: "rpz"
%%% End: 
