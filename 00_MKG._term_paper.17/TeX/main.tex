%% Преамбула TeX-файла

% 1. Стиль и язык
\documentclass[utf8x]{G7-32} % Стиль (по умолчанию будет 14pt)
\usepackage[T2A]{fontenc}
\usepackage[russian]{babel}
% Остальные стандартные настройки убраны в preamble.inc.tex.
\include{preamble.inc}

% Настройки листингов.
\include{listings.inc}

% Полезные макросы листингов.
\include{macros.inc}

\begin{document}

\frontmatter % выключает нумерацию ВСЕГО; здесь начинаются ненумерованные главы: реферат, введение, глоссарий, сокращения и прочее.

% Команды \breakingbeforechapters и \nonbreakingbeforechapters
% управляют разрывом страницы перед главами.
% По-умолчанию страница разрывается.

% \nobreakingbeforechapters
% \breakingbeforechapters

%\include{00-abstract}

\tableofcontents

%$tex

\Introduction

Целью данной работы является миссиологический анализ миссионерских проповедей Нового Завета на предмет особенности проповеди, зависивших от религиозной принадлежности адресатов. Для достижения поставленной цели необходимо:

\begin{itemize}
\item провести анализ миссионерских проповедей в Новом Завете: выявить внешний контекст и повлиявшие на проповедь факторы, структуру проповеди, внутреннюю логику, основное смысловое содержание, непосредственную реакцию на проповедь и возможные причины для этой реакции;
\item выделить в ходе анализа особенности, связанные с принадлежностью адресатов к иудаизму или язычеству
\end{itemize}

В качестве основы для миссиологического анализа проповедей, используется следующее определение миссионерского служения: "главное дело миссионера в его служении – возвещение людям в историческом времени керигмы Церкви и призыв к Покаянию через осознания себя и мира сего в бедственном положении в силу противоречия между предчувствием своего высшего призвания в прекраснейшем мире и реальностью господства зла в нём и в себе" (\cite{@ogk.mag.}, с. 66).
В данной работе мы рассматриваем и анализируем как основное содержание проповеди керигму, свидетельство о призвании человека и падшести мира, логику призыва к Покаянию.


Несмотря на вариативность миссионерских ситуаций и проповедников в Новом Завете, можно говорить, что "Новозаветные тексты... формируют единство в своём провозвестии единого Евангелия" (\cite{@dodd.apostolic}, с. 74).
Существуют определённые элементы "первоначальной керигмы", которые дают нам возможность называть проповедь миссионерской, т.е. благовестием \footnote{Эти понятия близки вплоть до тавтологии, т.к. "евангелия", "благие вести" – один из видов керигмы в современном апостолам государственном мире имперского Рима (см. подробнее \cite{@via.kerigma}}.


В исследовании для нас также значимо значение керигмы, подчёркнутое \cite{@bultman.theology_1} (с. 307): керигма – это "личное обращение... оно ставит человека перед вопросом... требуя от него решения".
Керигма апостольской миссионерской проповеди ожидает отклика, как его ждал от слушателей Иисус в Своей проповеди, и потому всегда есть "призыв к послушанию" \cite{@via.kerigma}.
В синоптических Евангелиях этот призыв резюмирован "двумя словами: кайтесь, веруйте" \cite{@dunn.edinstvo}.
Поэтому для нас важно наличие этого призыва в проповеди, а также непосредственная реакция слушателей во время проповеди и после.


Актуальность работы для современной миссиологии связана с тем, что "современное миссионерское служение Церкви основывается на двухтысячелетнем опыте православного свидетельства святоотеческой традиции"\cite{@rpc.concepcia} (31, с. 15).
Опыт апостольской проповеди интересен для нас не только как исторический, но и как практический.
Современный опыт христианского свидетельства отличается контекстом, но керигма не изменилась.
Что же касается опыта её применения в различных условиях, для этого многообразие новозаветной проповеди при единстве Благовестия для нас особенно ценно.
Наконец, при отсутствии ветхозаветных иудеев или язычников эллинистического толка в наше время, мы сталкиваемся с фундаменталистскими и секулярными течениями в самом христианстве, а также людьми вовсе не знакомыми с Писанием и Преданием церкви и открыто исповедающими служение языческим ценностям.
То, как подобным слушателям возвещалось христианство в 1 веке, актуально и для современной практики христианской миссии.


За основной источник миссиологического исследования Новозаветной проповеди мы берём книгу Деяний апостолов.
В книге Деяний изложены события после рождения Церкви, когда уже прозвучал призыв Иисуса проповедовать (ср. Мф 28:19, Лк 24:46-49, Мк 16:15, Ин 15:27).
\cite{@brown.vvedenie_1} называет Деян 1:8 "божественным планом миссионерства (10.2.3).
В книге описано множество ситуаций миссионерской проповеди, но таких мест, где, хотя бы частично, присутствует её текст, всего 12.
Среди них – 8 проповедей к иудеям и боящимся Бога из (бывших) язычников, 4 – к язычникам, не знающим Бога.
Их анализу посвящена данная работа.


\mainmatter % это включает нумерацию глав и секций в документе ниже


\chapter{ Особенности миссионерской проповеди иудеям и богобоязненным}
\label{cha:judes}
%
% % В начале раздела  можно напомнить его цель
%
Среди иудеев были не только евреи, но и обратившиеся в иудаизм из других народов: прозелиты, богобоязненные (не обрезавшиеся, но соблюдавшие Закон). Потому мы рассматриваем в качестве миссии иудеям проповедь ап Филиппа эфиопскому евнуху и проповедь ап Петра сотнику Корнилию, "мужу праведному и боящемуся Бога" с домочадцами.

\subsubsection*{Проповедь ап Петра в день Пятидесятницы (Деян 2:14-2:47)}
%%внешний контекст
После сошествия Св Духа на апостолов, они "начали говорить иными языками" (2:4), что привлекло внимание окружающих.
Так как был большой праздник Шавуот, в Иерусалиме присутствовало множество народу, как местных, так и приезжих "Иудеев, благоговейных людей из всякого народа под небом".
Слух о странном событии, случившимся с апостолами, собрал толпу, "пришедших в смятение", "изумлявшихся" и "дивившихся", т.к. каждый слышал, что апостолы говорили на его наречье.
К этим разноплемённым, но исповедавшим одну ветхозаветную веру людям, вышел со словом проповеди апостол Пётр.

Браун называет эту проповедь "фундаментальной формулировкой благовестия" в Деян (\cite{@brown.vvedenie_1}, 10.2.2).
В ней формируется керигма апостольской проповеди, отличной от проповеди Иисуса христологическим содержанием и исповеданием Иисуса воскресшим, Господом, Мессией, Сыном Божьим. 

%%структура проповеди и внутр. логика
\begin{center}
	\begin{longtable}{ |c|c|p{0.70\textwidth}| } 
 \hline
 стих & смысл & содержание \\ 
 \hline\hline
 2:16-21 & напоминание & пророчества Иоиля о сошествии Духа \\ 
 2:22-24 & керигма & предведение Божье об Иисусе, убийство и воскресение \\ 
 2:25-28 & напоминание & пророчество царя Давида о воскресении \\
 2:29-31 & толкование & Давид говорил о Христе \\
 2:32 & керигма & "Этого Иисуса воскресил Бог, чему все мы свидетели" \\
 2:33 & керигма & Иисус излил Духа на апостолов \\
 2:34-35 & напоминание & Давид пророчествовал о Господе \\
 2:36 & керигма & обличение: дом Израилев распял Господа и Христа \\
 2:38 & призыв & покатесь, креститесь, примите дар Святого Духа \\
 2:39 & керигма & обещание Божье – для всех, в т.ч. "дальних" \\ 
 \hline
\end{longtable}
\end{center}


В ходе проповеди, люди начали задавать вопросы ("Что нам делать, мужи братья?" 2:37), и возможно, вопросы не прекратились после, так как и "иными многими словами" апостол "свидетельствовал и увещал их".
И многие ("около трёх тысяч") крестились и вошли в общение и учение апостолов.

%%керигма
Керигма проповеди ап Петра: 
\begin{itemize}
 \item исповедание Иисуса Христом, о котором "предвидел Бог" и говорили пророки
 \item Иисуса воскресил Бог
 \item Иисус принял Духа от Бога и ниспослал Его на апостолов
 \item Иисус был распят
 \item Благая Весть теперь – для всех
\end{itemize}

Каждое возвещение апостол сопровождает ссылкой на Ветхий Завет: о Мессии и воскресении пророчествовал пророк Давид, о сошествии Духа – пророк Иоиль, обещание для "дальних" напоминает Ис 57:19,
а обличение "дома Израилева" в распятии Христа созвучно пророческим обличениям.
Апостол цитирует писание пророков и истолковывает, как пророчества раскрываются в Иисусе Христе.
Кроме того, что это подкрепляет его речь авторитетом Писания, а в слушателях – пробуждает память об уже воспринятом откровении,
толкование ап Петра раскрывает ранее непонятные и смутные смыслы о преодолении "тления" и наступлении "того Дня".
Он толкует то, о чём никто не мог говорить уверенно, и при том возвещает, что происходит это сейчас, у всех на глазах.

%%св-во о ч-ке и мире
Свидетельство о мире и человеке принимает образ эсхатологического исполнения: как в "тот День" (Иоиль 3:18), люди примут дар Святого Духа и получит отпущение грехов.
%% призыв к Покаянию
Это предваряется условием: покаянием и крещением (погружением) "во имя Иисуса Христа".

\subsubsection*{Проповедь апп Петра и Иоанна по исцелению хромого у Храма (Деян 3:1-4:4)}
%%внешний контекст
Апостолы продолжали ходить в храм для молитвы и теперь также для проповеди.
На ступенях они повстречали калеку, который не мог быть в храме из-за телесного изъяна, но просил милостыню у ворот.
В ответ на просьбу о милостыне, ап Пётр исцелил его от хромоты, и тот вошёл в храм, "ходя и скача и хваля Бога".
Бывшие в храме люди "исполнились ужаса и изумления" и "сбежался к ним (апостолам и исцелённому) в трепете весь народ в притворе, называемом Соломоновым".
К этим благочестивым иудеям обратил проповедь ап Пётр.

%%структура проповеди и внутр. логика
\begin{center}
	\begin{longtable}{ |c|c|p{0.55\textwidth}| } 
		\hline
		 стих & смысл & содержание \\
		   \hline\hline
		   3:12 & обращение & это чудо – не собственной силой апостолов \\
		   3:13a & керигма & Бог прославил Иисуса, "Отрока Своего" \\
		     3:13b-15a & керигма & вы отреклись и убили Начальника жизни \\
		     3:15b & керигма & Бог воздвиг Иисуса из мёртвых \\
		     3:16 & свидетельство & вера во имя Иисуса исцелила хромого \\
		      3:17 & обращение & вы отреклись по неведению \\
		      3:18 & напоминание & пророки предрекли страдания Христа \\
		       3:19-20 & призыв & "покайтесь и обратитесь" \\
		       3:21 & керигма, напоминание & Иисус на небе "до времён восстановления всего" \\
		       3:22-23 & напоминание & Моисей обещал о воздвижении Пророка "из братьев ваших" \\
		       3:24 & напоминание & Самуил и другие пророки "возвестили эти дни" \\
		       3:25 & напоминание & "вы сыны пророков и завета" \\
		       3:26 & керигма & Бог воскресил Иисуса и послал Его к каждому из вас, чтобы "отвратить от злых дел ваших" \\
		\hline
	\end{longtable}
\end{center}

%%керигма
Керигма проповеди апостола Петра:
\begin{itemize}
	\item Бог прославил Иисуса
	\item Иисус – "отрок" Божий
	\item Иисус был убит
	\item Бог воздвиг Иисуса из мёртвых
	\item Иисус придёт во "времена восстановления всего"
	\item Иисус послан к каждому
\end{itemize}

Помимо возвещения, апостол свидетельствует о чуде исцеления хромого "верой во имя Иисуса", обращает несколько раз внимание слушателей на самих себя, "мужей израильских", "братьев", к которым обращён призыв "Бога отцов ваших", и вспоминает пророчество Моисея, Самуила и "других пророков" об "этих днях" чудес.

%%св-во о ч-ке и мире
Свидетельство о мире и человеке опять звучит в эсхатологическом ключе: дни, о которых сказано у пророков, наступили, и, следовательно, чудеса не для изумления, а для "покаяния и обращения" к Сыну Божьему, о Котором все пророчества.

%% призыв к Покаянию
Апостол не успевает завершить проповедь призывом, так как "пока они говорили к народу, приступили к ним... и наложили на них руки и отдали под стражу" (4:1-3)
Однако, в 3:19-20 уже прозвучал призыв к покаянию, в 3:23 апостол предупреждает не слушающегося "Пророка Того", а в 3:26 говорит, что "Отрок" Божий послан Богом, "чтобы каждого отвратить от злых дел ваших".
Многих свидетельство исцеления и слово проповеди привело в Церковь (3:4).  


\subsubsection*{Проповедь апп Петра и Иоанна перед первосвященниками (Деян 4:5-21; 5:27-33)}
%%внешний контекст
После того, как апостолы были схвачены, собрался синедрион для решения по их вопросу, и их стали доправшивать, "какой силой или каким именем вы это сделали?".
Апостол Пётр, "исполнившись Духа Святого", начал проповедь перед первосвященниками, старейшинами и книжниками Израиля.


Увидев, что апостолов не переубедить, их изгнали с угрозами.
На следующий день они были вновь схвачены учащими в храме, и были подвергнуты новому допросу.
Апостол Пётр продолжил проповедь.

%%структура проповеди и внутр. логика
\begin{center}
	\begin{longtable}{ |c|c|p{0.70\textwidth}| } 
		\hline
		стих & смысл & содержание \\
		\hline\hline
		4:8-9 & обращение & вы допрашивайте тех, кто спас человека \\
		4:10 & свидетельство & хромой исцелён именем Иисуса Христа \\
		4:10b & керигма & Бог воздвиг Иисуса из мёртвых \\
		4:11-12 & керигма & вне Иисуса нет спасения \\
		4:19 & свидетельство & апостолы не могут не слушать Бога и не говорить о том, "что видели и слышали" \\
		5:29 & свидетельство & смысловой повтор 4:19 \\
		5:30 & обращение & обличение: вы умертвили Иисуса \\
		5:30-31 & керигма & Бог воздвиг Иисуса из мёртвых \\
		5:31b & керигма & Иисус – Начальник и Спаситель, в Котором возможно покаяние и отпущение грехов Израиля \\
		5:32 & керигма & Бог дал Духа Святого "повинующимся Ему" \\
		\hline
	\end{longtable}
\end{center}

Апостол Пётр в каждое обращение к слушающим вкладывает обличение: это они распяли Иисуса, они нуждаются в покаянии.
То, что они "дети Авраама", а Бог – "Бог их отцов", только усугбляет ситуацию, так как они погубили Того, Кого избрал и воздвиг от смерти Бог.



%%керигма
Керигма проповеди апостола Петра:
\begin{itemize}
	\item Бог воздвиг Иисуса из мёртвых
	\item вне Иисуса нет спасения
	\item в Иисусе возможно покаяние и отпущение грехов
	\item Бог даёт Духа "повинующимся Ему"
\end{itemize}

%%св-во о ч-ке и мире
Свидетельство о человеке и мире раскрывается в Иисусе: человеку необходимо обрести спасение, которое есть только "в имени Иисуса".

%% призыв к Покаянию
В слове апостола нет прямого призыва к обращению, он завуалирован в параллели между словами обличения и возвещением покаяния и отпущения грехов в Иисусе, Спасителе.

По итогам проповеди, апостолы едва не были осуждены на смерть, но по ходатайству фарисея и уважаемого законоучителя Гамалиила, были отпущены после бичевания.


\subsubsection*{Проповедь Стефана (Деян 7:2-60)}
%%внешний контекст
Дьякон Стефан, "полный благодати и силы", стал известен благодаря чудесам и знамениям.
Когда члены синагоги либертинцев, киринейцев и александрийцев затеяли с ним спор, им не удалось взять верх, и были найдены лжесвидетели, оклеветавшие юношу.
Перед судом Синедриона, после обвинений в хуле на Закон и Моисея, Стефану было дано слово и он начал проповедовать.
Браун упоминает, что ни одна из речей апостолов в Деян не приводится столь тщательно разработанной (в рамках написания текста) как речь Стефана (\cite{@brown.vvedenie_1}, 10.2.2).
В ней настолько "нестандартное понимание Ветхого Завета", что она стала предметом споров о внешнем влиянии на иудейскую традицию (там же).

%%структура проповеди и внутр. логика
\begin{center}
	\begin{longtable}{ |c|c|p{0.55\textwidth}| } 
		\hline
		стих & смысл & содержание \\
		\hline\hline
		7:2-47 & напоминание, обращение & напоминание действия Бога в Священной истории от Авраама до Соломона \\
		7:37 & напоминание & пророчество Моисея о воздвижении "пророка из братьев ваших, как меня" \\
		7:39 & обращение & обличение: "<Моисею> не захотели покориться отцы ваши" \\
		7:48-50 & напоминание & Ис 66:1, нет рукотворного дома для Бога Вседержителя \\
		7:51-53 & обращение & обличение: "вы всегда Духу Святому противитесь, как отцы ваши" \\
		7:55-56 & керигма, свидетельство & небеса отверсты, Сын Человеческий стоит по правую сторону Бога \\
		7:59 & свидетельство & молитва: "Господи Иисусе, прими дух мой" \\
		7:60 & свидетельство & ходатайственная молитва за убивающих его \\
		\hline
	\end{longtable}
\end{center}

%%керигма
Керигма проповеди Стефана:
\begin{itemize}
	\item небеса отверсты (что также может быть свидетельством о собственном пророческом даре от Духа
	\item Сын Человеческий стоит по правую сторону Бога
\end{itemize}

В проповеди Стефан, как делал Сам Иисус, не говорит прямо о Нём, но ясно даёт Его образ в служении Моисея "который принял слова живые, чтобы дать вам".
Он обличает слушателей в отвержении Моисея "которому не захотели покоиться отцы ваши" и всего обетования Божьего, данного ещё Аврааму.
А уже в этой борьбе с пророками совершается отвержение "Праведного, Которого предателями и убийцами вы теперь сделались".

%%св-во о ч-ке и мире
Стефан свидетельствует как пророк, его слова об "отверстых небесах" имеет параллель в пророческом тексте: "отверзлись небеса, и я видел видения Божии" (Иез 1:1, Син.).
А параллель с Дан 7:13-14 ("с облаками небесными шёл как бы Сын Человеческий... и Ему дана власть, слава и царство... и Царство Его не разрушится") приводит к мысли о свершившимся обетовании Бога, наступлении конца времени и Царства Бога.
Стефан свидетельствует о человеке и мире, что наступило время последнего суда и последнего выбора человека быть с Богом или отвергнуть Его.
Небо уже открыто и ждать какого-то ещё исполнения пророчеств не следует.

%% призыв к Покаянию
В словах Стефана нет прямого призыва к Покаянию, подобного призывам в проповедях ап Петра.
Есть сочетание обличения, свидетельства об исполнении обетования Бога и личная ходатайственная молитва за своих слушателей и губителей, что можно назвать личным покаянием за их грехи.

Прямое следствие проповеди – мученическая смерть Стефана.
Но, существует мнение, что его свидетельство сыграло значительную роль в истории Савла.
Так, Браун пишет "как смерть Иисуса не была Его концом, поскольку апостолы обрели Его Дух... так и смерть Стефана не была его концом, ибо её видит юноша по имени Савл... ему суждено продолжить дело Стефана" (\cite{@brown.vvedenie_1}, 10.2.2).


\subsubsection*{Проповедь ап Филиппа эфиопскому евнуху (Деян 8:26-40)}
%%внешний контекст
Апостол Филипп по повелению ангела идёт по пустынной дороге "от Иерусалима к Газе", и встречает Эфиопа, евнуха, вельможу Эфиопской царицы.
Эфиоп читал пророка Исайю и был на поклонении в Иерусалиме, откуда возвращался.
По велению ангела, ап Филипп начинает с ним разговор, после которого он крестится во имя Иисуса.

%%структура проповеди и внутр. логика
\begin{center}
	\begin{longtable}{ |c|c|p{0.55\textwidth}| } 
		\hline
		стих & смысл & содержание \\
		\hline\hline
		8:31 & обращение & понимает ли Эфиоп, что читает из пророка Исайи \\
		8:35 & напоминание, керигма & "начав от этого писания, благовествовал ему Иисуса" \\
		8:37 & керигма & евнух может креститься, если верует от всего сердца \\
		\hline
	\end{longtable}
\end{center}

%%керигма
В тексте Деян не приведена проповедь ап Филиппа, поэтому мы не можем сказать точно о её керигматической части.
Как минимум, это слово об исполнении писаний и пророчеств в Иисусе, а также универсальность откровения Нового Завета: возможность и необходимость крещения в Иисуса для каждого верующего "от всего сердца", независимо от национальных или физиологических предпосылок (ср. с запретом во Втор 23:1).
В этой универсальности – и новое откровение о человеке и мире: наступило время исполнения пророчеств, и каждый может и должен обратиться сердцем к Богу и сделать конкретный шаг: креститься. \footnote{У пророков уже звучало это сочетание эсхатологического времени и возможности иноплеменника и евнуха войти в дом и храм Божий (ср. Ис 56:3-5)}


По итогу проповеди, евнух сам спрашивает о возможности креститься и принимает крещение от ап Филиппа.
После ухода Филиппа, он "продолжал, радуясь, свой путь".

\subsubsection*{Проповедь ап Петра сотнику Корнилию с домочадцами и друзьями (Деян 10:24-48)}
%%внешний контекст
Эта проповедь продолжает "изменение правил", заложенное в крещении евнуха: Божье слово обращено к людям не из народа Израиля.
Хотя речь идёт о людях, знающих Закон, а именно "благочестивом и боящимся Бога со всем домом своим... молящимся Богу постоянно" сотнике Корнилии, они по-прежнему имеют статус "внешних".
Это оказывается столь сильным препятствием, что и для сотника, и для ап Петра требуется прямое Божье откровение, чтобы сделать шаг навстречу друг другу.

Получив видение от Бога о снятии запрета на нечистую пищу, ап Пётр преодолевает свои сомнения и приходит вслед за вестниками от Корнилия в его дом.
Корнилий созвал "родственников своих и ближайших друзей", чтобы "послушать речей" апостола

%%структура проповеди и внутр. логика
\begin{center}
	\begin{longtable}{ |c|c|p{0.55\textwidth}| } 
		\hline
		стих & смысл & содержание \\
		\hline\hline
		10:26 & свидетельство & "сам я тоже человек" \\
		10:28-29 & свидетельство & Бог указал не называть человека нечистым, "потому я пришёл" \\
		10:34-35 & керигма & Бог обращается к каждому "боящемуся Его и делающему правду" \\
		10:36 & керигма & Иисус Христос – "Господь всех" \\
		10:37-38 & напоминание & проповедь Иоанна Крестителя, жизнь Иисуса из Назарета \\
		10:38 & керигма & Иисус исполнял пророчества "благотворя и исцеляя", "Бог был с Ним" \\
		10:39 & керигма & Иисус был убит на кресте \\
		10:40-41 & керигма & Бог воздвиг Иисуса из мёртвых \\
		10:39-42 & свидетельство & мы – свидетели Его проповеди и воскресения \\
		10:42 & керигма & Иисус – "поставленный Богом Судия живых и мёртвых" \\
		10:43 & керигма, напоминание & свидетельство пророков – о Нём; в Нём – отпущение грехов верующим \\
		10:47 & свидетельство & Дух Святой сошёл на язычников, а значит и в крещении им нельзя отказать \\
		10:48 & призыв & повеление креститься \\
		\hline
	\end{longtable}
\end{center}

%%керигма
Керигма проповеди ап Петра:
\begin{itemize}
	\item Божье откровение универсально и обращено к каждому "боящемуся Его и делающему правду"
	\item Иисус Христос – Господь
	\item в Иисусе исполнились пророчества
	\item Иисус был убит на кресте
	\item Бог воздвиг Иисуса из мёртвых
	\item Иисус – Судия живых и мёртвых
	\item в Иисусе – отпущение грехов и спасение человека
	\item теперь Дух Святой сходит на людей, даже не принадлежащих народу Израиля по крови
\end{itemize}

%%св-во о ч-ке и мире
Апостол Пётр приносит новое свидетельство о человеке, основанное на собственном, новом для себя откровении.
Человек не может быть назван "нечистым" и не может быть отвергнут Богом по какому-либо внешнему признаку.
Дух Святой ниспослан Богом для каждого верующего во имя Иисуса.
Помимо факта прихода к "язычникам" и свидетельства о сошествии на них Духа, ап Пётр говорит о себе "сам я тоже человек", никак не выделяя ни своего апостольства, ни принадлежности израильскому народу.

Ещё одно важное свидетельство – Иисус – "Судия живых и мёртвых", т.е. свидетельство о Суде, на который предстанет (или уже предстаёт) каждый человек.

%% призыв к Покаянию
Призыв апостола креститься звучит уже в ответ на сошествие Духа Святого на Корнилия с друзьями и домочадцами.

\subsubsection*{Проповедь ап Павла в Антиохии Писидийской (Деян 13:14-44)}
%%внешний контекст
Апостол Павел со спутниками по повелению Духа Святого, путешествовали и проповедовали в Иудейских синагогах.
В Писидийской Антиохии в синагоге их пригласили сказать "слово наставления к народу" после чтения Закона и Пророков, и ап Павел стал говорить.

%%структура проповеди и внутр. логика
\begin{center}
	\begin{longtable}{ |c|c|p{0.70\textwidth}| } 
		\hline
		стих & смысл & содержание \\
		\hline\hline
		13:16 & обращение & "мужи Израильские и боящиеся Бога, послушайте" \\
		13:17-22 & напоминание & рассказ о Священной истории от Египетского плена до царя Давида \\
		13:23 & керигма & к Израилю пришёл Иисус Спаситель, сын Давида \\
		13:24-25 & напоминание & свидетельство Иоанна Крестителя об Иисусе \\
		13:26 & обращение & "нам послано слово спасения этого" \\
		13:27-29 & керигма & Иисус был отвергнут "начальниками" и "живущими в Иерусалиме" и распят \\
		13:30 & керигма & Бог воздвиг Иисуса из мёртвых \\
		13:31 & свидетельство & мы и многие другие – свидетели Его воскресения \\
		13:32-37 & керигма & в Иисусе исполняются пророчества \\
		13:33-35 & напоминание & Пс 2:7, "милости, обещанные Давиду" (ср. Иез 37:24, Ос 3:5), Пс 88:49 \\
		13:38-39 & керигма & во Христе – отпущение грехов и Спасение, которого не может быть в Законе \\
		13:40-41 & призыв & покайтесь ("презрители, подивитесь и сгиньте") \\
		13:41 & керигма & наступил день, когда "дело делаю Я (Господь)" \\
		\hline
	\end{longtable}
\end{center}

%%керигма
Керигма апостола Павла:
\begin{itemize}
	\item пришёл Иисус, Спаситель
	\item Иисус – сын Давида
	\item в Иисусе сбываются пророчества
	\item наступили дни Господа, возвещённые в пророчествах
	\item Иисуса отвергли и распяли
	\item Бог воздвиг Иисуса из мёртвых
	\item в Иисусе Христе возможно прощение грехов, невозможное по Закону
\end{itemize}

%%св-во о ч-ке и мире
Апостол Павел свидетельствует, что человек может быть прощён и спасён, если примет Иисуса Христа.
Мир находится в решающей стадии: пророчества сбываются, Господь "дело делает" своё, и для каждого встаёт вопрос о покаянии.

%% призыв к Покаянию
Свой призыв апостол Павел произносит словами, "сказанными у Пророков": "Посмотрите, презрители, подивитесь и сгиньте, потому что дело делаю Я в дни ваши, дело, которому вы никак не поверите, если кто расскажет вам" (ср. Ис 5:24, 28:14-15)

По итогам проповеди, апостола со спутниками просили проповедовать в следующую субботу, а многие последовали за ними и начали научаться.
Слухи разошлись и в следующую субботу "почти весь город собрался слушать слово Божие".

\subsubsection*{Проповедь ап Павла перед обвинителями в Иерусалиме (Деян 22:1-24)}
%%внешний контекст
В Иерусалиме, в храме, Апостол Павел был обвинён, якобы он осквернил храм введением в него язычников.
Едва не растерзанный толпой, он был взят под стражу римским трибуном с воинами, но попросил о возможности принести перед обвинителями оправдательную речь.
Трибун дал такую возможность.

%%структура проповеди и внутр. логика
\begin{center}
	\begin{longtable}{ |c|c|p{0.70\textwidth}| } 
		\hline
		стих & смысл & содержание \\
		\hline\hline
		22:1 & обращение & – \\
		22:3-5 & свидетельство & жизнь ап Павла до обращения: наставленный в Законе и гонитель христиан \\ 
		22:6-21 & свидетельство & обращение, исцеление Павла и послушание от Бога идти к язычникам \\
		22:8 & керигма & Иисус Назорей – Господь \\
		22:16 & керигма & крестясь и "призвав имя Его" человек "смывает" грехи \\
		22:21 & керигма & Бог посылает на проповедь к язычникам \\
		\hline
	\end{longtable}
\end{center}

%%керигма
Апостол Павел говорит в своё оправдание и больше приносит личное свидетельство, прямо и честно рассказывает о себе, проводя параллели между собой и слушающими ("как все вы сегодня" Деян 22:3).
Но, так как собственную жизнь он строит на Божьем откровении, его свидетельство в себе косвенно заключает керигму (как бы к нему самому обращённую):
\begin{itemize}
	\item Иисус – Господь
	\item в Иисусе – Спасение и прощение грехов
	\item откровение Божье теперь и для язычников
\end{itemize} 

%%св-во о ч-ке и мире
Свидетельство о человеке прежде всего в универсализме Божьего Слова, теперь обращённого ко всем народам, а также в возможности прощения грехов в крещении и "призвании имени" Иисуса Христа.

%% призыв к Покаянию
Апостол Павел не успевает призвать слушающих к покаянию, но приносит собственное свидетельство покаяния, в т.ч. публично признавая, что "сочувствовал и стерёг одежды убивавших" Стефана.

\subsection*{Сравнительный анализ миссионерских проповедей иудеям и "богобоязненным"}

В проповеди иудеям значительную роль играет возвещение исполнения ветхозаветных пророчеств.
Как для самих апостолов, так и для их окружения, выросшего в среде, где Закон определял всю общественную жизнь, а Пророки возвещали основные духовные чаяния, свидетельство о Христе в этих текстах играло значительную роль.
Их отношение к тексту, при этом, было довольно свободным, в свете уже состоявшейся жизни, смерти и воскресения Христа.
Так, Данн прибегает к термину цитаты-"пешер", цитаты-толкования, возникавшей "когда сводили воедино данность текста и данность евангельской традиции" (\cite{@dunn.edinstvo} § 24).
Такие "новозаветные цитаты из Ветхого Завета суть интерпретации", а не "независимый авторитет" (там же).
Опираясь на это свидетельство и толкуя их в свете Евангелия, апостолы проповедуют исполнение обетованного Богом, исполнение пророчеств, продолжая проповедь Христа о "приблизившемся Царстве".


В 6 из 8 проповедей проповедник прямо ссылается на пророческие тексты, и говорит об их исполнении.
В Антиохии Писидийской ап Павел говорит о Христе как "сыне Давида" и ссылается на проповедь «современного» ему пророка Иоанна Крестителя, а перед толпой гонителей в Деян 22 самого себя, воспитанного в Законе "у ног Гамалиила", ставит свидетелем исполнения ветхозаветных обетований и надежд.
Эти толкования отличаются от толкований "книжников" и звучат "со властью", так как исповедуют как уже случившееся то, что в писаниях только ожидалось.


Стефан исповедует перед смертью, что "Небеса отверсты" и вспоминает видения апокалиптических глав пророка Даниила.
В Антиохии ап Павел говорит об исполнении "милостей, обещанных Давиду" (Деян 13:33) и о том, что ныне "Дело делаю Я (Господь)" (13:41).
В проповеди после сошествия Духа Святого, апостол Пётр толкует это явление словами пророка Иоиля (и, в свою очередь, толкует пророка этим исполнением у всех на глазах).
После исцеления хромого у ворот храма, апостол Пётр говорит, что нет нужды удивляться чудесам, так как наступило время чудес, которое возвещали Пророки.
Это удивительное единодушие апостольского толкования пророческих текстов подражает эсхатологичности проповеди Самого Иисуса.


Эсхатологическая весть для знавших Писание и живших в среде, полной апокалиптических настроений и проповедей (напр. проповедь Иоанна Крестителя), разворачивало людей к знакомым словам Писания, а слова – раскрывала по-новому для слушателей (подобно тому как делал Иисус, напр. в Лк 24:12-35).
Пророчество о Дне Господа, о наступлении Царства раскрывалось как уже сбывающееся.
Уже в проповеди Иоанна Крестителя зазвучала эта нота незамедлительности наступления Царства, и перед слушающим вставал вопрос: что в связи с этим нужно делать\footnote{Который они озвучивали прямо во время проповеди, судя по: Лк 3:10-14, Деян 2:37, 9:6, 16:30}?


На это у апостолов есть ответ, что также отличает их от иудейских апокалиптиков: "В то время как чаяния иудейской эсхатологии были неопределенными или выраженными чисто символическим языком, апокалиптическая надежда христианства сосредоточивалась на конкретном человеке" (\cite{@dunn.edinstvo}, § 69), Иисусе Христе.
Иисус исповедуется как Судья, Господь и Спаситель, а в Деян 3:13 – "отрок" Божий.
В Нём возвещается спасение Израиля и отпущение грехов, которые невозможно простить по Закону (Деян 13:38-39, 22:16).


Есть ещё один аспект проповеди иудеям со ссылкой на ветхозаветные писания.
Это обличение, что особенно раскрывается в толковании Стефаном священной истории как истории непослушания и отступничества, вплоть до нарушения воли Божьей строительством рукотворного храма.
У других проповедников обличение не столь радикально, но отговорка "отец у нас – Авраам" (Лк 3:8) обращена против иудеев.
Они не узнали Христа, прославленного Богом, Спасителя, и распяли его.
Потому все чаяния их веры теперь безнадёжны, если они не крестятся во имя Иисуса, ведь в Нём исполняется их обетование.


Наряду со смертью Иисуса, в 6 проповедях сказано о воскресении (в разговорах с Корнилием и евнухом Эфиопом этот эпизод не приведён).
Это – ещё одно прямое возвещение того, о чём в словах пророках только упомянается и что едва ли поддавалось внятному толкованию вне победы Иисуса Христа над смертью.


В эпизодах с "боящимися Бога" Эфиопом и Корнилием с домочадцами и друзьями есть так же возвещение универсальности Божьего призыва и крещения во имя Иисуса.
Вера – не для определённого народа или физической чистоты, но от чистоты сердца "боящихся Его (Бога) и делающих правду" (Деян 10:34-35).
Обретающие таким образом Христа как Господа получают прощение своих грехов и освобождение от них и дары Святого Духа (Деян 10:47).


%%% Local Variables:
%%% mode: latex
%%% TeX-master: "rpz"
%%% End:



%&tex
\chapter{Миссиологический анализ проповедей иудеям}
\label{cha:hellenes}

В этом разделе рассмотрены 4 проповеди: в Листрах, в Афинах, перед прокуратором Феликсом, а также перед царём Агриппой \footnote{Хотя Агриппа II и был по происхождению иудей, он рос в Риме при императорском дворце и был скорее римского воспитания, чем иудейского. Его сестра, Друзилла, упомянутая в Деян 24, открыто перешла в язычество и была замужем за прокуратором Феликсом.}.

\section{Миссиологический анализ проповедей иудеям}
\subsubsection*{Проповедь ап Павла в Листрах (Деян 14:8-18)}
% контекст
После гонений со стороны иконийских иудеев, ап Павел со спутниками бежал в Ликаонские города.
В городе Листра ап Павел исцелил хромого от рождения человека, что вызвало своеобразное религиозное пробуждение в местных жителях.
Решив, что перед ними – ходящие по земле боги, они собрались принести им жертву.
Остановив их, ап Павел и Варнава обратились к ним с проповедью.

%%структура проповеди и внутр. логика
\begin{center}
	\begin{tabular}{ |c|c|c| }
		\hline
		стих & смысл & содержание \\
		\hline\hline
		14:15a & обращение, свидетельство & апостолы – не боги, а люди, несущие благую весть \\
		14:15b & керигма & человек должен обратиться от "суетных богов" к Богу живому, Творцу \\
		14:16-17 & керигма & Бог живой свидетельствовал о Себе людям добром и радостью, оставляя людям свободу \\
		14:18 & призыв & не приносить жертвы "суетным богам" \\
		\hline
	\end{tabular}
\end{center}

%%керигма
Керигма в проповеди ап Павла и Варнавы:
\begin{itemize}
	\item человек должен отказаться от идолов ради служения одному Богу живому
	\item истинный Бог – Творец мира
	\item от Бога – всё доброе и радость в жизни всех людей
	\item Бог оставляет человеку свободу "ходить своими путями"
	\item Бог "свидетельствует о Себе" и призывает к Себе
\end{itemize}

%%св-во о ч-ке и мире
Мир был сотворён Богом, и всё доброе в мире – от Бога. 
Идолы – "суетные боги", которым не следует поклоняться.
Человек свободен выбирать свой путь.

%% призыв к Покаянию
В словах проповеди, изложенных в Деян 14:8-18 не звучит прямого призыва, но из приведённых слов и фразы "успокоили народ, чтобы не приносили им жертвы", можно заключить наличие призыва оставить идолопоклонство.

%%реакция на проповедь
Несмотря на клевету на апостола и Варнаву со стороны иудеев, в Листре и близлежащих городах появились ученики и общины, которые они после наставляли (Деян 21-23).

\subsubsection*{Проповедь ап Павла в Ареопаге (Деян 17:16-34)}

% контекст
Апостол Павел ожидал спутников в Афинах и "дух в нём возмущался, видя, что город полон идолов".
Помимо обычной проповеди в синагоге, он начал разговаривать "на полащди каждый день со случайными встречными".
Философы и другие, кто "ничем другим не заполняли свои досуги как тем, чтобы говорить или слушать что-нибудь новое", заинтересовались и попросили рассказать в Ареопаге об этом "новом учении".

%%структура проповеди и внутр. логика
\begin{center}
	\begin{tabular}{ |c|c|c| }
		\hline
		стих & смысл & содержание \\
		\hline\hline
		17:22 & обращение & "по всему вижу, что вы особенно богобоязненны" \\
		17:23 & напоминание & жертвенник "неведомому богу" \\
		17:24 & керигма & Бог Творец и Господь неба и земли не живёт в рукотворённых храмах \\
		17:25 & керигма & Бог – источник жизни \\
		17:26 & керигма & Бог сотворил людей и поставил "сроки и пределы их обитанию" \\
		17:27a & керигма & призвание человека – искать Бога \\
		17:27b & керигма & Бог "не далеко от каждого из нас" \\
		17:28a & керигма & Бог – источник бытия и движения \\
		17:28b & напоминание & "как и некоторые из ваших поэтов сказали: «Ведь мы Его и род»" \\
		17:29 & керигма & Бог не подобен идолу, сделанному человеком \\
		17:30 & керигма, призыв & Бог "теперь" возвещает людям, "всем и всюду", покаяние \\
		17:31a & керигма & будет день Суда вселенной "по праведности" \\
		17:32b & керигма & Суд – через "поставленного" Им Мужа, воскрешённого из мёртвых \\
		\hline
	\end{tabular}
\end{center}


%%керигма
Керигма проповеди ап Павла:
\begin{itemize}
	\item единый истинный Бог – Творец и Господь неба и земли
	\item Бог – начало всего: жизни, бытия, движения
	\item Бог не подобен идолам и созданиям человеческой мысли и не нуждается в рукотворённых храмах
	\item человек сотворён Богом и призван Его искать
	\item Бог "не далеко" от каждого человека
	\item Бог лично призывает каждого человека покаяться
	\item грядёт Суд над вселенной "по праведности"
	\item Бог избрал Мужа, Который будет судить и Которого воскресил из мёртвых
\end{itemize}

%%св-во о ч-ке и мире
Апостол свидетельствует о сотворённости и конечности мира, а также о том, что Суд над миром будет "по праведности".
Человек сотворён Богом, Бог всегда рядом, человек призван Его найти и прийти к Нему, ответив на Его призыв покаянием.
В "Муже, Которого Он поставил" побеждена смерть.

%% призыв к Покаянию
Апостол объявляет, что Бог лично изменил ход истории: времена неведения прошли, и Бог теперь возвещает "всем и всюду" необходимость покаяния.
Покаяние нужно, так как Суд над миром – "по праведности".

%%реакция на проповедь
Слова о воскресении мёртвых вызывают непонимание и многие начинают насмехаться, хотя кое-кто просит послушать об этом ещё раз.
Некоторые из слушавших уверовали (Деян 17:34).

\subsubsection*{Проповедь ап Павла прокуратору Феликсу (Деян 24:10-26)}
% контекст
Обвиняемый первосвященником Ананией и другими старейшинами в осквернении храма, ап Павел предстал перед судом прокуратора Феликса.
Когда апостолу позволили говорить в свою защиту, он произнёс проповедь.

%%структура проповеди и внутр. логика
\begin{center}
	\begin{tabular}{ |c|c|c| }
		\hline
		стих & смысл & содержание \\
		\hline\hline
		24:10 & обращение & – \\
		24:11-13 & апология & Павел не совершил ничего дурного \\
		24:14-15 & свидетельство & "служу Богу отцов наших... надежду имея на Бога" \\ 
		24:14 & керигма & есть Путь служения Богу, согласно с Законом и Пророками \\
		24:15 & керигма & будет воскресение праведных и неправедных \\
		24:16 & свидетельство & потому апостол старается жить в непорочной совести \\
		24:17-20 & апология & Павел не совершил ничего дурного \\
		24:21 & свидетельство & ап Павел верует в воскресение мёртвых и готов на том стоять \\
		24:25 & призыв & призыв к "праведности" и "обладанию собой", т.к. будет Суд \\
		\hline
	\end{tabular}
\end{center}

%%керигма
Керигма в проповеди ап Павла:
\begin{itemize}
	\item существует Путь служения Богу
	\item грядёт воскресение праведных и неправедных
	\item требуется жить в "непорочной совести пред Богом и людьми", праведно, обладая собой
\end{itemize}


%%св-во о ч-ке и мире
Апостол свидетельствует, что всякого человека ожидает воскресение, и от всякого требуется праведность и чистота совести.
Что же касается воскресения Христа, то до личной встречи с Феликсом и Друзиллой (24:24-26), апостол не говорит об этом.
Возможно, причина в том, чтобы проповедь изабвления от смерти не оказалась "смазана" оттенком внутренних распрей иудеев о неких спорных духовных авторитетах.

%% призыв к Покаянию
После 24:21 Феликс, "имея более точные сведения о Пути", отослал первосвященника и обвинителей и поместил ап Павла под стражу, чтобы через несколько дней прийти к нему с женой Друзиллой, Иудеянкой и "слушать его о вере во Христа Иисуса".
Тогда ап Павел продолжил проповедь призывом к покаянию и праведной жизни.
Из Деян 24:26 следует, что безуспешно, так как прокуратор Феликс испугался слов о будущем суде и "надеялся, что Павел даст ему денег".


\subsubsection*{Проповедь ап Павла перед царём Агриппой (Деян 26:1-31)}
% контекст
Не желая, чтобы его отдали в руки иудеев, способных убить его, ап Павел потребовал суда Кесаря (императора), что позволялось римскому гражданину под угрозой смертного приговора.
По приезду царя Агриппы, апостолу было позволено держать защитительную речь перед ним, "трибунами и виднейшими в городе людьми", и он произнёс проповедь. 

%%структура проповеди и внутр. логика
\begin{center}
	\begin{tabular}{ |c|c|c| }
		\hline
		стих & смысл & содержание \\
		\hline\hline
		26:2-3 & обращение & царь Агриппа назван "знатоком всех обычаев и спорных мнений Иудеев" \\
		26:4-22 & свидетельство & история воспитания, обращения, исцеления и служения апостола \\
		26:6-8 & напоминание & Иудеи, "усердно служа Богу", надеются на обещание от Бога \\
		26:15 & керигма & Иисус – Господь \\
		26:17 & керигма & Иисус имеет слово для язычников \\
		26:18 & керигма, призыв & необходимо обратиться "от тьмы к свету и от власти сатаны к Богу" \\
		26:18b & керигма & отпущение грехов возможно в вере в Иисуса \\
		26:20 & призыв & апостол возвещал язычникам каяться и творить "достойные покаяния" дела \\
		26:23 & керигма & Христос пострадал, но воскрес из мёртвых \\
		26:23b & керигма & проповедь Христа – для всех народов \\
		26:25 & апология & проповедь апостола – "слова истины и здравого смысла" \\
		26:26 & призыв & если Агриппа верит Пророкам, ему надо поверить в проповедь ап Павла \\
		26:29 & свидетельство & апостол надеется, что все уверуют во Христа \\
		\hline
	\end{tabular}
\end{center}

%%керигма
Керигма в проповеди ап Павла:
\begin{itemize}
	\item Иисус – Господь
	\item проповедь Христа – для всех народов
	\item во Христе возможно отпущение грехов
	\item Христос пострадал, но воскрес из мёртвых
	\item необходимо освобождение от власти сатаны и обращение к Богу, покаяние и творение "достойных покаяния дел"
\end{itemize}

%%св-во о ч-ке и мире
Апостол свидетельствует, что человек призван к свободе от власти греха и сатаны.
Проповедь Христа возвещает всем людям освобождение и отпущение грехов в вере в Него.
Обетование и надежда Иудеев – на воскресение, и это сбывается во Христе.

%% призыв к Покаянию
Апостол прямо обращается к царю Агриппе с призывом откликнуться на его проповедь, если он верит словам Пророков.
Как надлежит откликнуться – он говорит ранее, рассказывая о язычниках, которые покаялись и теперь творят "достояные покаяния дела".

%%реакция на проповедь
Неизвестно, какими были плоды проповеди кроме того, что Агриппа и его приближённые нашли, "что ничего достойного смерти или уз человек этот не делает".


\section{Особенности миссионерской проповеди язычникам в Новом Завете}
Основной особенностью проповеди язычникам стала необходимость проповедовать Единого Бога Творца и призывать к отказу от идолопоклонства.
В двух проповедях (Листры и Афины) присутствует это свидетельство, разное по форме, но единое в содержании.
Во всех случаях Бог исповедуется как:  
\begin{itemize}
	\item Творец и начало всего
	\item единый Бог для всех, а не для какого-либо народа
	\item Живой Бог, лично принимающий участие в жизни человека, с Которым возможны личные отношения
	\item истинный, отличающийся от созданий человеческих рук и мыслей (идолов)
	\item имеющий благое попечение о человеке и человечестве
\end{itemize}

В разговоре с простыми людьми (судя по всему, Листра – небольшой город), апостол ссылается не на философов и поэтов, а на действие Божье в жизни его слушателей: "подавая вам с неба дожди и времена плодоносные, исполняя пищею и радостью сердца ваши". 
Бог как пища, радость, дождь – един для всех и для каждого, близок и благ к каждому.
Кроме того, Он "позволил всем народам ходить своими путями", а значит лично принимает свободу человека и ждёт таких же личных отношений от него.


В Листрах Бог возвещается как создатель "неба и земли, и моря и всего, что в них", – как практически детским перечислением (ср. Пс 103) творения исповедуется Бог, как начало всего.
Совсем по-другому – в Ареопаге: "Сам дарует всем жизнь и дыхание и всё" (начало жизни), "произвёл род человеческий... предуставив сроки и пределы их обитанию", "«Ведь мы Его и род»" (начало людей), "в Нём мы живём и движемся и существуем" (начало бытия и движения).
Бог – конечный ответ на поиск "начал" в философии и личности человека (Отец, хотя бы как основатель рода человеческого).
И во всём – с акцентом на личностности Бога: "Бог не далеко от каждого из нас", "произвёл... род человеческий... искать Бога", – т.е., так же оставил свободу выбора пути, хоть и с личной заинтересованностью в близости.


Таким образом, в откровении о Едином Боге говорится и откровение о человеке как свободном, но призванном к духовному пути близости к своему Творцу.
Но человек ещё только призван стать таким, обратиться "от тьмы к свету и от власти сатаны к Богу" (Деян 26:18), получить отпущение грехов.
Как и в Лк, в Деян Царство Божье проповедуется как избавление человека от власти духов мира сего, от власти болезней и немощей и, в итоге, смерти (чему соответствует проповедь воскресения Христова как исполнения обетованного Пророками).


Эта свобода актуализируется в оставлении служения "руками человеческими" созданиям "печати искусства и мысли человеческой" в "рукотворённых храмах".
В проповедях перед Ареопагом, перед прокуратором Феликсом и перед царём Агриппой апостол наряду с возвещением свободы говорит о суде и покаянии, которые отличаются от привычных методов оценки человеческой жизни божественными силами (либо равнодушными, либо смотрящими на богатство приносимых даров).
Суд над вселенной будет "по праведности" (Деян 17:31), язычники, которым проповедовал ап Павел "обращались к Богу, творя дела, достойные покаяния" (Деян 26:18), и с прокуратором Феликсом апостол начинает говорить о необходимости изменения жизни: "обладании собой", "праведности" (Деян 24:24-25).
В Листрах же ап Павел и Варнава буквально силой останавливают языческую церемонию, которой хотели отметить местные жители чудо исцеления хромого.
Бог проповедуется как имеющий всё и, соответственно, не "имеющий в чём-либо нужду" от человека, но призывающий человека к себе через покаяние (обращение) (Деян 17:30).


Этот призыв, как исповедует апостол, теперь обращён ко "всем и всюду", потому что проходят "времена неведения", и определён день суда (Деян 17:30-31). 
Эсхатологический элемент керигмы, пробуждающий актуальную память знавшего слова пророков, для язычников был приниципиально новым свидетельством о мире, времени, истории\footnote{См. напр. \cite{@dunn.vvedenie}: "Апокалиптическая эсхатология принимает еврейское видение истории (практически уникальное для античности): история мыслится линейно... как движение вперёд к определённому концу и цели" (§ 66.3)}.
Без этого элемента возвещения Суда и призыв к покаянию как возможности для Спасения, были бы невозможны.


Интересно, что проповедуя в Ареопаге, апостол Павел отталкивается от архаического (вероятно, даже в то время) образа-мифа "неведомого бога", а не от представления о мире одной из популярных в то время философских школ.
Судя по цитатам, он был знаком с культурой и взглядами на устройство мира эллинистического мира.
Но, обращается к глубоко религиозному чувству трепета перед неведомым, тем отметая возможность расценить его речь как приглашение к философскому дискурсу.
Эта проповедь "не в убедительных словах человеческой мудрости" (1Кор 2:4) вызывает непонимание одних ("одни насмехались"), но интерес других ("мы послушаем тебя об этом ещё раз").


В рамках того же религиозного трепета перед неведомым, ап Павел в проповеди язычникам аппелирует к столь же неразрешимым загадкам сил природы и смерти.
Если по первому вопросу в иудейском мире ещё существовал ответ: надо поклоняться не силам, а Богу, от Которого всё произошло, то последний ставил в тупик и иудеев, и эллинов.
Его разрешение было возможно только в керигме Господа над всем, включая смерть, от которой Его избавляет Бог.


Апостол проповедовал то, что, вероятно, когда-то "пронзило" его самого, ведь "для самого апостола Павла важнейшим эсхатологическим событием, в котором живой Бог открыл... Свой замысел о спасении мироздания, стало \textit{воскресение Иисуса из мёртвых}"\cite{@wright.4to}.
Победа над смертью означала победу над грехом, а значит – полное освобождение от власти сатаны, полный и окончательный переход под власть Бога.
Отсюда – постоянство керигмы о воскресении Христа, несмотря на неготовность слушателей эллинистического воспитания.


%%% Local Variables:
%%% mode: latex
%%% TeX-master: "rpz"
%%% End:



\backmatter %% Здесь заканчивается нумерованная часть документа и начинаются ссылки и
            %% заключение

%$tex

\Conclusion % заключение к отчёту

В ходе анализа 12 миссионерских проповедий книги Деяний, были выделены структура, основное керигматическое содержание, логика призыва к Покаянию, свидетельство о человеке и мире в свете Новозаветного откровения.
Также, удалось выявить особенности проповеди для слушателей, живущих в иудейском Законе и знающих Бога Живого и для язычников.


Общей для всех случаев керигмой стало возвещение Христа, убитого и воскресшего, Господа и Судии, в Котором – освобождение от грехов.
Особенность проповеди Христа сказывается в том, что иудеям Его смерть и воскресение возвещаются как обличение, делающее недопустимым надежду на спасение в Законе и Пророках: народ Израиля не узнал и отверг Мессию и Спасителя.
Язычникам же Христос не всегда и не сразу возвещается прямо, не произносится и Его имя и происхождение, вероятно, чтобы не затушевать универсализм спасительной вести национальным оттенком.


Второе общее для всех случаев возвещение – близость Царства Божьего.
Все миссионерские проповеди Нового Завета носят эсхатологический оттенок исполнения пророчеств, наступления долгожданного Дня Господня, близости суда Христа, приходящего в силе.
И для иудеев, и для язычников, такая проповедь ставит вопрос радикально: прямо сейчас (или уже никогда) следует принять решение вверить себя Христу.
Основные формы такого возвещещения: интерпритационная проповедь Пророков как исполняющейся на глазах слушателей реальности, возвещение грядущего Суда, сам жанр керигмы-евангелия\footnote{Будучи формулой возвещения императорской власти, керигма в римском юридическом понимании не предполагала иной возможности кроме подчинения. Подробнее см. \cite{@via.kerigma}}.
Такое возвещение было испытанием и для самого проповедника, жизнь которого должна была соответствовать вести о приблизившемся Царстве.
Может быть, одна из причин, по которой эсхатологизм в современной миссионерской проповеди (в отличии от апокалиптизма, подробнее см. \cite{@dunn.vvedenie § 69} редок, и проповедь может не иметь той "соли" апостольской.


Эсхатологичность и проповедь Христа как Господа и Спасителя логически приводят к вопросу о дальнейших шагах.
Большинство проанализированных проповедей заканчивались призывом к покаянию и крещению.
И покаяние, и крещение проповедовались как дело открытого Богу сердца и чистой совести.
Потому оно было возможно для евнуха или необрезанных из язычников, и оставалось необходимым для "сынов Авраама", исполняющих Закон.


Основное различие между проповедью иудеям и язычником связано с необходимостью проповеди для последних Единого Живого Бога Творца и отречению от идолов ради Него.
Апостолы нигде не пропускали этого этапа проповеди и явно чувствовали разницу в проповеди язычникам.
Известные миссионеры ХХ века, такие как К. С. Льюис или митр. Антоний (Блум) хорошо чувствовали необходимость проповеди Живого Бога Отца миру, незнакомому с Ним, и находили слова для этого, понятные современному им человеку.
Первыми эту традицию проповеди среди язычников заложили апостолы, вышедшие с Благой Вестью за пределы Израиля.


Для любого современного миссионера, открывающего книгу Деяний, встанет вопрос о том, как обрести силу духа и слова апостолов.
Анализ проповедей показывает нам, как эти люди в разных ситуациях, перед разъярённой толпой или перед сельскими жителями поражёнными чудесами, перед собравшимися на досуг афинянами или на суде, подбирали слова, способные донести до слушателя ту керигму, которой жили сами.
Их собственное соответствие их возвещению позволило слову достигать сердца людей, и из их проповеди родилась христианская церковь.

%%% Local Variables: 
%%% mode: latex
%%% TeX-master: "rpz"
%%% End: 



\input{81-biblio}

\appendix   % Тут идут приложения

\chapter{Проповеди иудеям}
\label{cha:appendix1}

\subsection*{Проповедь ап Петра в день Пятидесятницы (Дян 2:14-2:47)}
14 Петр же, став с одиннадцатью, возвысил свой голос и возгласил им: мужи Иудейские и все, живущие в Иерусалиме! Да будет вам это известно и вслушайтесь в слова мои.

15 Ибо не пьяны они, как вы предполагаете: ведь только третий час дня.

16 Но это то, что сказано чрез пророка Иоиля:

17 "И будет в последние дни, говорит Бог: изолью от Духа Моего на всякую плоть, и будут пророчествовать сыны ваши и дочери ваши, и юноши ваши будут видеть видения. И старцам вашим будут сниться сны,

18 и на рабов Моих и на рабынь Моих в те дни изолью от Духа Моего, и будут пророчествовать.

19 И дам чудеса на небе вверху и знамения на земле внизу, кровь и огонь и клубы дыма.

20 Солнце превратится в тьму, и луна в кровь, прежде чем придет день Господень великий и блистающий.

21 И будет: всякий, кто призовет имя Господне, будет спасен".

22 Мужи Израильские! Выслушайте эти слова: Иисуса Назорея, Мужа, Богом отмеченного для вас силами, и чудесами, и знамениями, которые сотворил чрез Него Бог среди вас, как вы сами знаете,

23 Его, по определению и предведению Божию преданного, вы, пригвоздив рукою беззаконных, убили.

24 Бог Его воскресил, разрешив муки смерти, потому что Он не мог быть держим ею.

25 Ибо Давид говорит о Нем: "Я видел Господа предо мною непрестанно, ибо Он по правую мою сторону, дабы я не поколебался.

26 Оттого возрадовалось мое сердце и возликовал язык мой, даже и плоть моя пребудет в надежде,

27 ибо Ты не оставишь души моей во аде и не дашь святому Твоему увидеть тление.

28 Ты поведал мне пути жизни, Ты исполнишь меня радости пред лицом Твоим".

29 Мужи братья! Да позволено будет сказать вам с дерзновением о патриархе Давиде, что он и скончался и погребен, и гробница его у нас до сего дня.

30 Итак, будучи пророком и зная, что клятвою поклялся ему Бог от плода чресл его посадить на престоле его,

31 - он, провидя, изрек о воскресении Христа, что и Он не оставлен во аде, и плоть Его не увидела тления.

32 Этого Иисуса воскресил Бог, чему все мы свидетели.

33 Итак, десницей Божией вознесённый, и приняв от Отца обещанного Духа Святого, Он излил то, что вы и видите и слышите.

34 Ибо Давид не восшел на небеса, но он сам говорит: "Сказал Господь Господу Моему: сядь по правую сторону Мою,

35 доколе Я не положу врагов Твоих в подножие ног Твоих".

36 Итак, твердо знай весь дом Израилев, что и Господом и Христом сделал Его Бог, Того Иисуса, Которого вы распяли.

37 Они же, услышав, поражены были в самое сердце, и сказали Петру и остальным апостолам: что нам делать, мужи братья?

38 А Петр им сказал: покайтесь, и да крестится каждый из вас во имя Иисуса Христа для отпущения грехов ваших, и вы получите дар Святого Духа.

39 Ибо для вас обещание и для детей ваших и для всех дальних, кого ни призовет Господь Бог наш.

40 И иными многими словами он свидетельствовал и увещал их, говоря: спасайтесь от этого извращенного рода.

41 Итак, приняв слово его, они были крещены. И присоединилось в день тот душ около трех тысяч.

42 И пребывали они постоянно в учении апостолов и в общении, в преломлении хлеба и в молитвах.

43 Был же во всякой душе страх. И много чудес и знамений совершалось чрез апостолов.

44 А все уверовавшие были вместе и всё у них было общее.

45 И то, чем владели и что имели, они продавали и разделяли это всем, смотря по нужде каждого.

46 И каждый день единодушно пребывая в храме, и преломляя по домам хлеб, они принимали пищу в веселии и простоте сердца,

47 хваля Бога и будучи в милости у всего народа. А Господь прибавлял к ним спасаемых каждый день.

\subsection*{Проповедь апп Петра и Иоанна перед первосвященниками (Деян 4:5-21)}

5 И было на следующий день, что собрались в Иерусалиме начальники их и старейшины и книжники,

6 и Анна первосвященник и Каиафа, и Иоанн и Александр, и все, кто были из первосвященнического рода,

7 и, поставив их посредине, допрашивали: какой силой или каким именем вы это сделали?

8 Тогда Петр, исполнившись Духа Святого, сказал им: начальники народа и старейшины,

9 если нам сегодня чинят допрос о благодеянии человеку больному, каким образом он спасен,

10 то да будет известно вам и всему народу Израильскому, что именем Иисуса Христа Назорея, Которого вы распяли, Которого Бог воздвиг из мертвых, – Им стоит он перед вами здоровый.

11 Он есть Камень, признанный негодным вами – строителями, оказавшийся во главе угла.

12 И нет ни в ком другом спасения. Ибо под небом нет и иного имени, данного людям, которым надлежит нам быть спасенными.

13 Видя же дерзновение Петра и Иоанна, и поняв, что они люди некнижные и простецы, удивлялись и признавали их, что они были с Иисусом.

14 И видя, что человек, исцеленный, с ними стоит, ничего не могли возразить.

15 И, приказав им выйти вон из синедриона, совещались друг с другом

16 и говорили: что нам делать с этими людьми? Ведь то, что ими совершено замечательное знамение, это явно всем живущим в Иерусалиме, и мы не можем отрицать;

17 но, чтобы оно еще больше не распространилось в народе, пригрозим им, чтобы они уже от этого имени не говорили никому из людей.

18 И, призвав их, приказали совсем не говорить и не учить во имя Иисуса.

19 Но Петр и Иоанн сказали им в ответ: рассудите, справедливо ли пред Богом, слушать вас больше чем Бога?

20 Ибо не можем мы не говорить о том, что видели и слышали.

21 И, снова пригрозив, отпустили их, так и не находя повода наказать их, из-за народа, потому что все прославляли Бога за происшедшее.

\subsubsection*{Продолжение проповеди (Деян 5:27-33)}
27 И приведя их, поставили в синедрионе. И спросил их первосвященник:

28 мы строжайше приказали вам не учить от этого имени. И вот, вы наполнили Иерусалим учением вашим и хотите навести на нас кровь Этого Человека.

29 Но Петр и апостолы сказали: повиноваться должно Богу больше чем людям,

30 Бог отцов наших воздвиг Иисуса, Которого вы умертвили, повесив на древе:

31 Его Бог вознес десницей Своей, как Начальника и Спасителя, чтобы дать Израилю покаяние и отпущение грехов.

32 И мы свидетели тому, и Дух Святой, Которого Бог дал повинующимся Ему.

33 Слышавшие были в бешенстве и хотели убить их.

\subsection*{Проповедь Стефана (Деян 7:2-60)}
2 И тот сказал: Мужи братья и отцы! Послушайте! Бог славы явился отцу нашему Аврааму, когда он был еще в Месопотамии, прежде поселения его в Харране, и сказал ему:

3 "выйди из земли твоей и из родни твоей и иди в землю, которую Я тебе укажу".

4 Тогда, выйдя из земли Халдейской, поселился он в Харране. И оттуда, по смерти отца его, Бог переселил его в эту землю, на которой вы теперь живете.

5 И не дал ему наследия в ней ни на стопу ноги, но обещал дать ее во владение ему и семени его после него, обещал, когда не было у него детей.

6 И Бог сказал так: "будет семя его пришельцем на земле чужой, и поработят его и будут творить ему зло четыреста лет.

7 И народ, которому они будут рабами, судить буду Я" - Бог сказал, - "и после этого они выйдут и будут служить Мне на этом месте".

8 И дал ему завет обрезания. И так родил он Исаака и обрезал его в день восьмой; и Исаак – Иакова, и Иаков – двенадцать патриархов.

9 И патриархи, позавидовав Иосифу, продали его в Египет. И был Бог с ним,

10 и избавил его от всех скорбей и дал ему милость и мудрость в глазах фараона, царя Египетского. И поставил его начальником над Египтом и над всем домом его.

11 И пришел голод на весь Египет и Ханаан, и скорбь великая; и не находили пищи отцы наши.

12 Иаков же, услышав, что есть хлеб в Египте, послал отцов наших в первый раз.

13 И во второй раз узнан был Иосиф братьями своими, и стал известен фараону род Иосифа.

14 Иосиф же послал и вызвал Иакова, отца своего, и всю родню: семьдесят пять душ.

15 И сошел Иаков в Египет и скончался сам и отцы наши;

16 и были они перенесены в Сихем и положены в гробнице, которую купил Авраам ценой серебра у сынов Емморовых в Сихеме.

17 Когда же стало приближаться время исполниться обещанию), которое Богу угодно было дать Аврааму, возрос народ и умножился в Египте.

18 Так было, пока не восстал иной царь над Египтом, который не знал Иосифа.

19 Он, прибегнув к хитрости против рода нашего, в злобе на отцов, заставлял их бросать младенцев своих, чтобы те не оставались в живых.

20 В это время родился Моисей и был прекрасен пред Богом. Его кормили три месяца в доме отца.

21 А когда был он брошен, взяла его к себе дочь фараона и воспитала его как своего сына.

22 И научен был Моисей всей мудрости Египетской и был силен в словах и делах своих.

23 Когда же исполнялось ему сорок лет, пришло ему на сердце посетить братьев своих, сынов Израилевых.

24 И увидев, что кого-то обижают, он заступился и отомстил за страдавшего, поразив Египтянина.

25 Он думал, что братья понимают, что Бог рукой его дает спасение им; но они не поняли.

26 На следующий день он появился между ними, когда они дрались, и склонял их к миру словами: "мужи, вы – братья; почему вы обижаете друг друга?"

27 Но обижающий ближнего оттолкнул его, сказав: "кто тебя поставил начальником и судьей над нами?

28 Не хочешь ли ты убить меня так же, как убил вчера Египтянина?"

29 И бежал Моисей из-за этого слова и стал пришельцем в земле Мадиамской, где он родил двух сыновей,

30 И по исполнении сорока лет, явился ему в пустыне горы Синая Ангел в пламени горящей купины.

31 Моисей же, увидев, дивился видению. А когда он подходил, чтобы всмотреться, раздался голос Господа:

32 "Я – Бог отцов твоих, Бог Авраама, и Исаака, и Иакова". Объятый трепетом, Моисей не посмел всматриваться,

33 И сказал ему Господь: "сними обувь с ног твоих, ибо место, на котором ты стоишь, земля святая.

34 Увидел Я, увидел притеснение народа Моего, который в Египте, и стенание его услышал и нисшел освободить их. И теперь иди, Я пошлю тебя в Египет".

35 Этого Моисея, которого они отвергли, сказав: "кто тебя поставил начальником и судьей?" - его-то послал Бог и начальником и избавителем, рукою Ангела, явившегося ему в купине.

36 Это он вывел их, творя чудеса и знамения в земле Египетской и в Чермном море и в пустыне сорок лет.

37 Это тот Моисей, который сказал сынам Израилевым: "Пророка вам воздвигнет Бог из братьев ваших, как меня".

38 Это тот, кто в собрании в пустыне был с Ангелом, говорившим ему на горе Синае, и с отцами нашими; тот, который принял слова живые, чтобы дать вам,

39 которому не захотели покориться отцы ваши, но отринули его и обратились сердцами своими к Египту,

40 сказав Аарону: "сделай нам богов, которые пойдут впереди нас, ибо Моисей этот, который вывел нас из земли Египетской, – не знаем, что случилось с ним".

41 И сделали они тельца в дни те, и принесли жертву идолу, и радовались делу рук своих.

42 И отвратился Бог и предал их служить воинству небесному, как написано в книге Пророков: "Заклания и жертвы принесли ли вы Мне за сорок лет в пустыне, дом Израилев?

43 И взяли вы с собою скинию Молоха и звезду бога вашего Ремфана, изображения, которые вы сделали, чтобы поклоняться им, и Я переселю вас дальше Вавилона".

44 Скиния свидетельства была у отцов наших в пустыне, как повелел Говоривший Моисею сделать её по образу, который тот увидел.

45 Её, получив по преемству, и внесли отцы наши с Иисусом во владения народов, которых изгнал Бог от лица отцов наших. Так было до дней Давида,

46 который обрел благодать пред Богом и молился о том, чтобы найти жилище Богу Иакова.

47 Соломон же воздвиг Ему дом.

48 Но не живет Всевышний в рукотворенном, как говорит пророк:

49 "Небо – Мне престол, а земля – подножие ног Моих. Какой дом вы построите Мне, говорит Господь, или какое место покоя Моего?

50 Не Моя ли рука сотворила это всё?"

51 Жестоковыйные и необрезанные сердцем и ушами, вы всегда Духу Святому противитесь, как отцы ваши, так и вы.

52 Кого из пророков не гнали отцы ваши? И убили они предвозвестивших пришествие Праведного, Которого предателями и убийцами вы теперь сделались,

53 вы, которые получили Закон в наставлениях ангельских и не сохранили.

54 Слыша это, они кипели бешенством в сердцах своих и скрежетали на него зубами.

55 А он, полный Духа Святого, устремил взор к небу и увидел славу Божию и Иисуса, стоящего по правую сторону Бога,

56 и сказал: вот я вижу отверстые небеса и Сына Человеческого стоящего по правую сторону Бога.

57 Они же, закричав громким голосом, зажали уши свои и устремились единодушно на него;

58 и, выгнав его за пределы города, побивали камнями. И свидетели сложили одежды свои у ног юноши, называемого Савлом.

59 И побивали камнями Стефана, а он взывал и говорил: Господи Иисусе, прими дух мой.

60 И склонив колени, воскликнул громким голосом: Господи, не вмени им этого греха! И сказав это, почил.

\subsection*{Проповедь ап Филиппа эфиопскому евнуху (Деян 8:26-40)}
26 А ангел Господень сказал Филиппу: встань и к полудню иди на дорогу, спускающуюся от Иерусалима к Газе. Она безлюдна.

27 И встав, он пошел. И вот муж Эфиоп, евнух, вельможа Кандакии, царицы Эфиопской, который был над всей ее казной. Он ходил для поклонения в Иерусалим,

28 и возвращался. И сидел он на своей колеснице и читал пророка Исаию.

29 И Дух сказал Филиппу: подойди и пристань к этой колеснице,

30 И подбежав, Филипп услышал, что он читает Исаию пророка,

31 и сказал: да понимаешь ли ты, что читаешь? А он сказал: как мне понять, если кто не наставит меня? И попросил Филиппа подняться и сесть с ним.

32 А место Писания, которое он читал, было такое: Как овца на заклание Он был приведен, и как агнец пред стригущим Его безгласен, так Он не отверзает уст Своих.

33 В унижении Его, было отказано Ему в правосудии. Род Его кто изъяснит? Ибо жизнь Его изъемлется от земли.

34 И обращаясь к Филиппу, евнух сказал: прошу тебя, о ком пророк говорит это: о себе или о ком-нибудь другом?

35 Филипп же отверз уста свои и, начав от этого писания, благовествовал ему Иисуса.

36 Двигались они по дороге и пришли к какой-то воде. И говорит евнух: вот вода; что препятствует мне креститься?

37 И сказал ему Филипп: если веруешь от всего сердца, можно. Он же сказал в ответ: верую, что Иисус Христос есть Сын Божий.

38 И приказал остановить колесницу, и сошли оба в воду, как Филипп, так и евнух, и он крестил его.

39 Когда же они вышли из воды, Дух Господень восхитил Филиппа, и больше не видел его евнух, и продолжал он, радуясь, свой путь.

40 А Филипп оказался в Азоте и, проходя, благовествовал всем городам, доколе не пришел в Кесарию.

\subsection*{Проповедь ап Петра сотнику Корнилию с домочадцами и друзьями (Деян 10:24-48)}
24 А на следующий день вошли они в Кесарию. Корнилий же ожидал их, созвав родственников своих и ближайших друзей.

25 А когда Петр готов был войти, Корнилий, встретив его, пал к его ногам и поклонился.

26 Но Петр поднял его говоря: встань, сам я тоже человек.

27 И беседуя с ним, вошел, и находит многих в сборе,

28 и сказал он им: вы знаете, как незаконно для Иудея сближаться с иноплеменником или приходить к нему; а мне Бог указал не называть ни одного человека скверным или нечистым.

29 Поэтому я и пришел беспрекословно, будучи позван. Итак, я спрашиваю: по какой причине вы послали за мной?

30 И Корнилий сказал: в этот час пошел четвертый день, как я, в девятый час, молился у себя в доме, и вот стал передо мною муж в одежде блестящей

31 и говорит: Корнилий, услышана твоя молитва, и милостыни твои воспомянуты пред Богом.

32 Итак, пошли в Иоппию за Симоном, который прозывается Петром: он гостит в доме Симона Кожевника близ моря".

33 Поэтому я тотчас же послал к тебе, и ты хорошо сделал, что пришел. Теперь же мы все пред Богом, чтобы выслушать всё, что повелено тебе Господом.

34 Петр отверз уста и сказал: Поистине я убеждаюсь, что Бог нелицеприятен,

35 но во всяком народе боящийся Его и делающий правду приятен Ему.

36 Он послал сынам Израилевым слово, благовествуя мир чрез Иисуса Христа: Этот есть Господь всех.

37 Вы знаете, что произошло по всей Иудее, начиная от Галилеи, после крещения, которое проповедал Иоанн:

38 об Иисусе из Назарета, как помазал Его Бог Духом Святым и силою, и Он ходил, благотворя и исцеляя всех угнетаемых диаволом, потому что Бог был с Ним.

39 И мы свидетели всего, что сделал Он в стране Иудейской и в Иерусалиме. Его они и убили, повесив на древе.

40 Его Бог воздвиг в третий день и дал Ему являться

41 – не всему народу, но свидетелям, предызбранным Богом, нам, которые с Ним ели и пили по воскресении Его из мертвых.

42 И Он повелел нам проповедать народу и засвидетельствовать, что Он есть поставленный Богом Судия живых и мертвых.

43 О Нем все пророки свидетельствуют, что всякий верующий в Него получит отпущение грехов именем Его.

44 Петр еще произносил эти слова, как сошел Дух Святой на всех слышащих слово.

45 И изумились верующие из обрезанных, которые пришли с Петром, что дар Святого Духа излился и на язычников;

46 ибо они слышали их, говорящих языками и величающих Бога.

47 Тогда ответил Петр: может ли кто отказать в воде крещения тем, кто приняли Духа Святого, как и мы?

48 И велел им креститься во имя Иисуса Христа. Тогда они попросили его остаться на несколько дней.

\subsection*{Проповедь ап Павла в Антиохии Писидийской (Деян 13:14-44)}
14 Они же, пройдя путь от Пергии, прибыли в Антиохию Писидийскую и, придя в синагогу в день субботний, сели.

15 И после чтения Закона и Пророков начальники синагоги послали им сказать: мужи братья, если у вас есть слово наставления к народу, говорите.

16 И Павел, встав и дав знак рукой, сказал: мужи Израильские и боящиеся Бога, послушайте.

17 Бог народа этого, Израиля, избрал отцов наших и возвысил народ во время пребывания в земле Египетской, и рукою высокою вывел их из нее,

18 и около сорока лет терпел их в пустыне

19 и, истребив семь народов в земле Ханаанской, дал им в наследие землю их

20 приблизительно на четыреста пятьдесят лет. И после этого дал судей до Самуила пророка.

21 И затем они просили царя, и дал им Бог Саула, сына Киса, мужа из колена Вениаминова, на сорок лет.

22 И отстранив его, Он воздвиг им в цари Давида, о котором и сказал во свидетельство: "Я нашел Давида, сына Иессея, мужа по сердцу Моему, который исполнит всю волю Мою".

23 От его-то семени Бог, по обещанию, привел Израилю Спасителя Иисуса

24 после того, как Иоанн проповедовал, перед самым явлением Его, крещение покаяния всему народу Израильскому.

25 А когда Иоанн завершил свое поприще, он говорил: "я не то, что вы обо мне думаете; но вот, идет за мною Тот, у Которого я недостоин развязать обувь на ногах".

26 Мужи братья, сыны рода Авраамова, и боящиеся Бога между вами: нам послано слово спасения этого.

27 Ибо живущие в Иерусалиме и начальники их, не узнав Его, исполнили и голоса пророков, читаемые каждую субботу, осудив Его.

28 И не найдя никакого основания для смерти, упросили Пилата убить Его.

29 Когда же исполнили всё написанное о Нем, то, сняв с древа, положили Его в гробницу,

30 Но Бог воздвиг Его из мертвых.

31 Он являлся в течение многих дней пришедшим вместе с Ним из Галилеи в Иерусалим. Они теперь свидетели Его перед народом.

32 И мы вам благовествуем обещание, которое дано было отцам,

33 что Бог его исполнил для нас, их детей, воскресив Иисуса, как и написано в псалме втором: "Ты – Сын Мой, Я сегодня родил Тебя".

34 А что Он воскресил Его из мертвых, так что Он уже не обратится в тление, – Он сказал так: "Я дам вам святое достояние Давидово непреложное".

35 Поэтому Он и в другом месте говорит: "Ты не дашь святому Твоему увидеть тление".

36 Но Давид, послужив в своем поколении совету Божию, почил и приложился к отцам своим и увидел тление.

37 Тот же, Кого Бог воздвиг, не увидел тления.

38 Итак, да будет известно вам, мужи братья, что ради Него вам возвещается отпущение грехов, и во всём, в чем вы не могли быть оправданы Законом Моисеевым,

39 всякий верующий оправдывается Им.

40 Итак, берегитесь, чтобы не пришло на вас сказанное у Пророков:

41 "Посмотрите, презрители, подивитесь и сгиньте, потому что дело делаю Я в дни ваши, дело, которому вы никак не поверите, если кто расскажет вам".

42 И когда они выходили, их просили, чтобы в следующую субботу им сказаны были эти слова.

43 И когда собрание было распущено, последовали многие из Иудеев и благоговейных прозелитов за Павлом и Варнавой, которые, беседуя с ними, убеждали их пребывать в благодати Божией.

44 А в следующую субботу почти весь город собрался слушать слово Божие.

\subsection*{Проповедь ап Павла перед обвинителями в Иерусалиме (Деян 22:1-24)}
1 Мужи братья и отцы, выслушайте мою нынешнюю перед вами защиту.

2 Услышав, что он начал свое обращение к ним на еврейском языке, они стали еще тише.

3 И он сказал: я – Иудей, рожденный в Тарсе Киликийском, но воспитанный в этом городе, у ног Гамалиила, наставленный во всей точности отеческого Закона, ревнитель по Боге, как все вы сегодня.

4 На этот Путь я воздвигнул смертельное гонение, заключая в узы и предавая в тюрьмы как мужчин, так и женщин,

5 в чем и первосвященник мне свидетель и весь совет старейшин; получив от них и письма к братьям, я шел в Дамаск с тем, чтобы и там находящихся привести в узах в Иерусалим для наказания.

6 И было со мной, когда я шел и приближался к Дамаску: около полудня внезапно воссиял с неба сильный свет вокруг меня,

7 и я упал на землю и услышал голос, говорящий мне: "Саул, Саул, что ты Меня гонишь?"

8 И я ответил: "кто Ты, Господи?" И Он сказал мне: "Я – Иисус Назорей, Которого ты гонишь".

9 Бывшие со мной свет видели, но голоса Того, Кто мне говорил, не слышали.

10 И я сказал: "что мне делать, Господи?" Господь же сказал мне: "встань и иди в Дамаск и там тебе будет сказано о всём, что назначено тебе делать".

11 И пока я ничего не видел от славы света того, бывшие со мной за руку привели меня в Дамаск.

12 А некий Анания, муж благочестивый по Закону, имеющий доброе свидетельство от всех местных Иудеев,

13 пришел ко мне и, подойдя, сказал мне: "Саул, брат, прозри!" И я в тот же час прозрел и устремил взор на него.

14 А он сказал: "Бог отцов наших предназначил тебя познать волю Его и увидеть Праведного и услышать голос из уст Его,

15 потому что ты будешь свидетелем Ему пред всеми людьми о том, что ты видел и слышал.

16 И теперь, что ты медлишь? Восстав, крестись и смой грехи твои, призвав имя Его".

17 И было со мной по возвращении в Иерусалим, когда я молился в храме: пришел я в исступление

18 и увидел Его, и Он говорил мне: "поспеши и выйди скорее из Иерусалима, потому что не примут твоего свидетельства о Мне".

19 сказал: "Господи, они знают, что я заключал в тюрьмы и бил в синагогах верующих в Тебя;

20 и когда проливалась кровь Стефана, свидетеля Твоего, я и сам стоял тут же и сочувствовал и стерег одежды убивавших его".

21 И Он сказал мне: "иди, потому что Я пошлю тебя далеко к язычникам".

22 До этого слова они его слушали, а тогда возвысили голос свой, говоря: долой с земли такого! Нельзя ему жить!

23 И пока они кричали и потрясали одеждами и бросали пыль в воздух,

24 трибун велел ввести его в казарму, сказав подвергнуть его допросу под бичем, чтобы узнать, по какой причине они так кричали на него.


%%% Local Variables: 
%%% mode: latex
%%% TeX-master: "rpz"
%%% End: 

\chapter{Проповеди язычникам}
\label{cha:appendix2}

\subsection*{Проповедь ап Павла в Листрах (Деян 14:8-18}
8 И некий муж в Листрах, не владевший ногами, сидел; хромой от чрева матери своей, он никогда не ходил.

9 Он слышал, как говорил Павел, который, устремив на него взор и увидев, что он имеет веру, чтобы быть спасенным,

10 сказал громким голосом: встань на ноги твои прямо. И он вскочил и стал ходить.

11 Народ же, увидев, что сделал Павел, возвысил свой голос, говоря по-ликаонски: боги в образе человеческом сошли к нам.

12 И называли они Варнаву Зевсом, а Павла Гермесом, так как он держал речь.

13 И жрец Зевса, стоящего перед городом, доставив к воротам быков и венки, хотел с народом принести жертву.

14 Но апостолы Варнава и Павел, услышав, разорвали одежды свои и с криком бросились в толпу,

15 говоря: мужи, что это вы делаете? И мы – подобные вам люди, благовествующие вам, чтобы вы от этих суетных богов обратились к Богу живому, Который сотворил небо и землю, и море и всё, что в них,

16 Который в прошедших поколениях позволил всем народам ходить своими путями,

17 хотя и не переставал свидетельствовать о Себе, творя добро, подавая вам с неба дожди и времена плодоносные, исполняя пищею и радостью сердца ваши.

18 И говоря это, они едва успокоили народ, чтобы не приносили им жертвы.

\subsection*{Проповедь ап Павла в Ареопаге (Деян 17:16-34)}
16 И пока Павел ожидал их в Афинах, дух его в нем возмущался, видя, что город полон идолов.

17 Итак, он рассуждал в синагоге с Иудеями и чтущими Бога, и на площади каждый день со случайными встречными.

18 А кое-кто и из эпикурейских и стоических философов встречался с ним, и некоторые говорили: что хочет сказать этот пустослов? А другие: кажется, это проповедник чужих богов (потому что он благовествовал Иисуса и воскресение).

19 И взяв его, привели в Ареопаг и говорили: можем ли мы узнать, что это за новое учение, проповедуемое тобой?

20 Ибо странное что-то вкладываешь ты в наши уши. Вот мы и хотим узнать, что это может быть?

21 Афиняне же все и живущие у них чужестранцы ничем другим не заполняли свои досуги, как тем, чтобы говорить или слушать что-нибудь новое.

22 И Павел, став посредине Ареопага, сказал: мужи Афиняне, по всему вижу, что вы особенно богобоязненны.

23 Ибо, проходя и осматривая ваши святыни, я нашел и жертвенник, на котором было написано: "неведомому богу". Итак, что вы, не зная, чтите, я возвещаю это вам.

24 Бог, сотворивший мир и всё, что в нём, Он, Господь неба и земли, не в рукотворенных храмах обитает,

25 и не руками человеческими воздается Ему служение, как имеющему в чем-либо нужду: Он Сам дарует всем жизнь и дыхание и всё.

26 И произвёл Он от одного весь род человеческий: обитать по всему лицу земли, предуставив сроки и пределы их обитанию;

27 искать Бога, не коснутся ли они Его и не найдут ли, хотя и не далеко Он от каждого из нас.

28 Ибо в Нем мы живем и движемся и существуем, как и некоторые из ваших поэтов сказали: "Ведь мы Его и род".

29 Итак, будучи родом Божиим, мы не должны думать, что Божество подобно золоту, или серебру, или камню, носящим печать искусства и мысли человеческой,

30 Поэтому, оставив без внимания времена неведения, Бог теперь возвещает людям, всем и всюду, чтобы они каялись,

31 ибо Он определил день, когда будет судить вселенную по праведности, чрез Мужа, Которого Он поставил, дав удостоверение всем, воскресив Его из мертвых.

32 Услышав же о воскресении мертвых, одни насмехались, другие сказали: мы послушаем тебя об этом еще раз.

33 Так Павел вышел из среды их.

34 Но некоторые люди, примкнув к нему, уверовали: между ними и Дионисий Ареопагит и женщина, по имени Дамарь, и другие с ними.

\subsection*{Проповедь ап Павла прокуратору Феликсу (Деян 24:10-26)}
10 И ответил Павел, когда правитель сделал ему знак говорить: зная, что ты с давних лет судишь этот народ, я с легким сердцем буду защищать мое дело,

11 так как ты можешь узнать, что – не более двенадцати дней, как я пришел для поклонения в Иерусалим.

12 И не нашли меня ни в храме с кем-либо спорящим или вызывающим волнение народа, ни в синагогах, ни в городе,

13 и не могут доказать тебе того, в чем теперь обвиняют меня.

14 Но в том я признаюсь тебе, что на Пути, который они называют ересью, я, действительно, служу Богу отцов наших, веруя всему, что согласно с Законом и что написано у Пророков,

15 надежду имея на Бога, которую и сами они разделяют, – что будет воскресение и праведных и неправедных.

16 Потому я и сам стараюсь всегда иметь непорочную совесть пред Богом и людьми.

17 Через несколько лет я прибыл с целью передать милостыни народу моему и приношения.

18 При этом нашли меня очистившимся в храме, не с толпой и не с шумом,

19 нашли же какие-то Иудеи из Асии, которым надлежало бы быть здесь у тебя и обвинять, если бы у них было что против меня.

20 Или пусть они сами скажут, какое нашли они преступление, когда я предстал пред синедрионом,

21 кроме одного этого слова, которое я возгласил, стоя между ними: "за воскресение мертвых вы меня судите сегодня".

22 Но Феликс, имея более точные сведения о Пути, отослал их для дополнительного расследования, сказав: когда придет трибун Лисий, я рассмотрю ваше дело;

23 и дал распоряжение сотнику держать его под стражей, но давать ему некоторые послабления, и не препятствовать никому из близких его служить ему.

24 Несколько дней спустя, Феликс, прибыв с Друзиллой, женой своей, Иудеянкой, послал за Павлом и слушал его о вере во Христа Иисуса.

25 Но так как он говорил о праведности и обладании собой и о будущем суде, то Феликс, придя в страх, ответил: пока иди, а при случае я тебя вызову к себе.

26 В то же время он и надеялся, что Павел даст ему денег. Потому-то, часто за ним посылая, он и беседовал с ним.

\subsection*{Проповедь ап Павла перед царём Агриппой (Деян 26:1-31)}
1 И сказал Агриппа Павлу: тебе разрешается говорить за себя. Тогда Павел, протянув руку, начал свою защитительную речь:

2 во всём, в чем обвиняют меня Иудеи, царь Агриппа, я почитаю себя счастливым защищаться сегодня перед тобой,

3 потому особенно, что ты знаток всех обычаев и спорных мнений Иудеев. Поэтому прошу выслушать меня с долготерпением.

4 Жизнь мою от юности, протекавшую с самого начала среди народа моего в Иерусалиме, ведают все Иудеи,

5 зная обо мне издавна, если есть у них желание свидетельствовать – что жил я фарисеем, по строжайшему направлению в нашей вере.

6 И теперь я стою перед судом за надежду на обещание, бывшее от Бога отцам нашим,

7 исполнения которого надеются достичь наши двенадцать колен, усердно служа Богу день и ночь. За эту надежду, царь, и обвиняют меня Иудеи.

8 Почему у вас считается невероятным, что Бог воздвигает мертвых?

9 Ведь я и сам думал, что против имени Иисуса Назорея надо мне многое сделать.

10 Это я и делал в Иерусалиме, и многих из святых я заключил в тюрьмы, получив власть от первосвященников, и когда их убивали, подавал свой голос против них;

11 и по всем синагогам, многократно наказывая их, принуждал к хуле и, в чрезмерной против них ярости, преследовал их даже и в чужих городах.

12 В этих условиях, отправляясь в Дамаск с полномочиями и поручением от первосвященников,

13 я в полдень на дороге увидел, царь, свет с неба, сильнее солнечного блеска, осиявший меня и шедших со мной.

14 И когда все мы упали на землю, я услышал голос, говорящий мне на еврейском языке: "Саул, Саул, что ты меня гонишь? Трудно тебе идти против рожна".

15 Я сказал: "Кто Ты, Господи?" Господь сказал: "Я Иисус, Которого ты гонишь.

16 Но встань и стань на ноги твои, ибо Я для того явился тебе, чтобы поставить тебя служителем и свидетелем Моим, как ты Меня видел, и как Я явлюсь тебе,

17 избавляя тебя от народа и от язычников, к которым Я посылаю тебя,

18 открыть им глаза, чтобы обратились они от тьмы к свету и от власти сатаны к Богу, и чтобы получили они отпущение грехов и удел вместе с освященными, по вере в Меня".

19 Поэтому, царь Агриппа, я не оказал непослушания небесному видению,

20 но сперва находящимся в Дамаске и Иерусалиме, и по всей стране Иудейской, и язычникам возвещал, чтобы они каялись и обращались к Богу, творя дела, достойные покаяния.

21 За это Иудеи, задержав меня в храме, пытались расправиться со мной.

22 Итак, с помощью, которая приходит от Бога, я устоял до сего дня, свидетельствуя малому и великому, не говоря ничего, кроме того, о чем Пророки сказали и Моисей, что должно тому быть:

23 Христу – пострадать и, воскресши первым из мертвых, свет возвещать народу и язычникам.

24 Когда он так защищался, Фест громким голосом говорит: ты безумствуешь, Павел! Большая ученость доводит тебя до безумия.

25 А Павел говорит: я не безумствую, превосходнейший Фест, но возглашаю слова истины и здравого смысла.

26 Ибо знает об этом царь, которому я и говорю с дерзновением. Ведь я не верю, чтобы сокрыто было от него что-либо из этого; ибо не в углу это было совершено.

27 Веришь ли, царь Агриппа, Пророкам? Знаю, что веришь.

28 Но Агриппа Павлу: еще немного, и ты будешь убеждать меня сделаться христианином.

29 А Павел: молил бы я Бога, чтобы мало ли, много ли, не только ты, но и все слушающие меня сегодня, сделались такими же как и я, кроме этих уз.

30 И встал царь и правитель и Вереника и сидевшие с ними,

31 и удалившись, говорили между собой, что ничего достойного смерти или уз человек этот не делает.

%%% Local Variables: 
%%% mode: latex
%%% TeX-master: "rpz"
%%% End: 



\end{document}

%%% Local Variables:
%%% mode: latex
%%% TeX-master: t
%%% End:

