%$tex

\Conclusion % заключение к отчёту

В ходе анализа 12 миссионерских проповедий книги Деяний, были выделены структура, основное керигматическое содержание, логика призыва к Покаянию, свидетельство о человеке и мире в свете Новозаветного откровения.
Также, удалось выявить особенности проповеди для слушателей, живущих в иудейском Законе и знающих Бога Живого и для язычников.


Общей для всех случаев керигмой стало возвещение Христа, убитого и воскресшего, Господа и Судии, в Котором – освобождение от грехов.
Особенность проповеди Христа сказывается в том, что иудеям Его смерть и воскресение возвещаются как обличение, делающее недопустимым надежду на спасение в Законе и Пророках: народ Израиля не узнал и отверг Мессию и Спасителя.
Язычникам же Христос не всегда и не сразу возвещается прямо, не произносится и Его имя и происхождение, вероятно, чтобы не затушевать универсализм спасительной вести национальным оттенком.


Второе общее для всех случаев возвещение – близость Царства Божьего.
Все миссионерские проповеди Нового Завета носят эсхатологический оттенок исполнения пророчеств, наступления долгожданного Дня Господня, близости суда Христа, приходящего в силе.
И для иудеев, и для язычников, такая проповедь ставит вопрос радикально: прямо сейчас (или уже никогда) следует принять решение вверить себя Христу.
Основные формы такого возвещещения: интерпритационная проповедь Пророков как исполняющейся на глазах слушателей реальности, возвещение грядущего Суда, сам жанр керигмы-евангелия\footnote{Будучи формулой возвещения императорской власти, керигма в римском юридическом понимании не предполагала иной возможности кроме подчинения. Подробнее см. \cite{@via.kerigma}}.
Такое возвещение было испытанием и для самого проповедника, жизнь которого должна была соответствовать вести о приблизившемся Царстве.
Может быть, одна из причин, по которой эсхатологизм в современной миссионерской проповеди (в отличии от апокалиптизма, подробнее см. \cite{@dunn.vvedenie § 69} редок, и проповедь может не иметь той "соли" апостольской.


Эсхатологичность и проповедь Христа как Господа и Спасителя логически приводят к вопросу о дальнейших шагах.
Большинство проанализированных проповедей заканчивались призывом к покаянию и крещению.
И покаяние, и крещение проповедовались как дело открытого Богу сердца и чистой совести.
Потому оно было возможно для евнуха или необрезанных из язычников, и оставалось необходимым для "сынов Авраама", исполняющих Закон.


Основное различие между проповедью иудеям и язычником связано с необходимостью проповеди для последних Единого Живого Бога Творца и отречению от идолов ради Него.
Апостолы нигде не пропускали этого этапа проповеди и явно чувствовали разницу в проповеди язычникам.
Известные миссионеры ХХ века, такие как К. С. Льюис или митр. Антоний (Блум) хорошо чувствовали необходимость проповеди Живого Бога Отца миру, незнакомому с Ним, и находили слова для этого, понятные современному им человеку.
Первыми эту традицию проповеди среди язычников заложили апостолы, вышедшие с Благой Вестью за пределы Израиля.


Для любого современного миссионера, открывающего книгу Деяний, встанет вопрос о том, как обрести силу духа и слова апостолов.
Анализ проповедей показывает нам, как эти люди в разных ситуациях, перед разъярённой толпой или перед сельскими жителями поражёнными чудесами, перед собравшимися на досуг афинянами или на суде, подбирали слова, способные донести до слушателя ту керигму, которой жили сами.
Их собственное соответствие их возвещению позволило слову достигать сердца людей, и из их проповеди родилась христианская церковь.

%%% Local Variables: 
%%% mode: latex
%%% TeX-master: "rpz"
%%% End: 

