%$tex

\Introduction

Целью данной работы является миссиологический анализ миссионерских проповедей Нового Завета на предмет особенности проповеди, зависивших от религиозной принадлежности адресатов. Для достижения поставленной цели необходимо:

\begin{itemize}
\item провести анализ миссионерских проповедей в Новом Завете: выявить внешний контекст и повлиявшие на проповедь факторы, структуру проповеди, внутреннюю логику, основное смысловое содержание, непосредственную реакцию на проповедь и возможные причины для этой реакции;
\item выделить в ходе анализа особенности, связанные с принадлежностью адресатов к иудаизму или язычеству
\end{itemize}

В качестве основы для миссиологического анализа проповедей, используется следующее определение миссионерского служения: "главное дело миссионера в его служении – возвещение людям в историческом времени керигмы Церкви и призыв к Покаянию через осознания себя и мира сего в бедственном положении в силу противоречия между предчувствием своего высшего призвания в прекраснейшем мире и реальностью господства зла в нём и в себе" (\cite{@ogk.mag.}, с. 66).
В данной работе мы рассматриваем и анализируем как основное содержание проповеди керигму, свидетельство о призвании человека и падшести мира, логику призыва к Покаянию.


Несмотря на вариативность миссионерских ситуаций и проповедников в Новом Завете, можно говорить, что "Новозаветные тексты... формируют единство в своём провозвестии единого Евангелия" (\cite{@dodd.apostolic}, с. 74).
Существуют определённые элементы "первоначальной керигмы", которые дают нам возможность называть проповедь миссионерской, т.е. благовестием \footnote{Эти понятия близки вплоть до тавтологии, т.к. "евангелия", "благие вести" – один из видов керигмы в современном апостолам государственном мире имперского Рима (см. подробнее \cite{@via.kerigma}}.


В исследовании для нас также значимо значение керигмы, подчёркнутое \cite{@bultman.theology_1} (с. 307): керигма – это "личное обращение... оно ставит человека перед вопросом... требуя от него решения".
Керигма апостольской миссионерской проповеди ожидает отклика, как его ждал от слушателей Иисус в Своей проповеди, и потому всегда есть "призыв к послушанию" \cite{@via.kerigma}.
В синоптических Евангелиях этот призыв резюмирован "двумя словами: кайтесь, веруйте" \cite{@dunn.edinstvo}.
Поэтому для нас важно наличие этого призыва в проповеди, а также непосредственная реакция слушателей во время проповеди и после.


Актуальность работы для современной миссиологии связана с тем, что "современное миссионерское служение Церкви основывается на двухтысячелетнем опыте православного свидетельства святоотеческой традиции"\cite{@rpc.concepcia} (31, с. 15).
Опыт апостольской проповеди интересен для нас не только как исторический, но и как практический.
Современный опыт христианского свидетельства отличается контекстом, но керигма не изменилась.
Что же касается опыта её применения в различных условиях, для этого многообразие новозаветной проповеди при единстве Благовестия для нас особенно ценно.
Наконец, при отсутствии ветхозаветных иудеев или язычников эллинистического толка в наше время, мы сталкиваемся с фундаменталистскими и секулярными течениями в самом христианстве, а также людьми вовсе не знакомыми с Писанием и Преданием церкви и открыто исповедающими служение языческим ценностям.
То, как подобным слушателям возвещалось христианство в 1 веке, актуально и для современной практики христианской миссии.


За основной источник миссиологического исследования Новозаветной проповеди мы берём книгу Деяний апостолов.
В книге Деяний изложены события после рождения Церкви, когда уже прозвучал призыв Иисуса проповедовать (ср. Мф 28:19, Лк 24:46-49, Мк 16:15, Ин 15:27).
\cite{@brown.vvedenie_1} называет Деян 1:8 "божественным планом миссионерства (10.2.3).
В книге описано множество ситуаций миссионерской проповеди, но таких мест, где, хотя бы частично, присутствует её текст, всего 12.
Среди них – 8 проповедей к иудеям и боящимся Бога из (бывших) язычников, 4 – к язычникам, не знающим Бога.
Их анализу посвящена данная работа.
